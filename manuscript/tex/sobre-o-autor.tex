\chapter{Sobre o Autor}

\enlargethispage{\baselineskip}

O Venerável Ajahn Sumedho foi o primeiro monge ocidental a ser ordenado na 
Tradição Tailandesa da Floresta, tradição esta que faz parte do Budismo Theravada, 
a escola do Budismo que predomina no Sri Lanka e no Sudeste Asiático.
Neste último século a clareza e simplicidade dos ensinamentos budistas têm sido
bem recebidos no Ocidente como fonte de paz e de compreensão, que faz frente
aos rigorosos testes da nossa presente geração.

Ajahn Sumedho nasceu em Seattle, Washington, em 1934. Cresceu no seio de uma
família anglicana juntamente com a sua irmã mais velha. Aos dezoito anos de idade
serviu na Marinha dos Estados Unidos, durante quatro anos, o que incluiu o
período da Guerra da Coreia. Após o serviço militar completou um bacharelato em
Estudos do Extremo Oriente, em 1973, graduando"-se com um mestrado em Estudos do
Sudoeste Asiático, na Universidade da Califórnia, Berkeley. Serviu no Corpo de
Paz como professor de Inglês de 1974 a 1976, trabalhando após esse período como
assistente social na Cruz Vermelha.

Desiludido e insatisfeito com o dogmatismo da religião ocidental, decide em 1966
viajar até à Tailândia para praticar meditação em Wat Mahathat, Bangkok. Pouco
depois em 1967 decide ordenar"-se como monge em Nong Khai, Nordeste da Tailândia.
Após um ano de prática solitária sentiu necessidade de um professor que pudesse
guiá"-lo mais profundamente. Foi então que ocorreu um afortunado encontro com um
monge visitante que o levou à província de Ubon para conhecer o Venerável
Ajahn~Chah,
no mosteiro Wat Nong Pah Pong. Tornou"-se então, oficialmente um discípulo
de Ajahn Chah e permaneceu sob a sua tutela durante dez anos. Os mosteiros de
Ajahn Chah eram conceituados pela sua austeridade e abordagem simples e directa
da prática do Dhamma. Em 1975 Ajahn Sumedho estabeleceu Wat Pah Nanachat,
Mosteiro Internacional da Floresta, com o objectivo de treinar os ocidentais
interessados em seguir a vida monástica.

Em 1977, a convite do \emph{English Sangha Trust}, acompanhou Ajahn Chah a
Inglaterra, residindo temporariamente na Hampstead Vihāra, em Londres, com
outros três monges. O objectivo do \emph{English Sangha Trust} era estabelecer
as condições apropriadas para o treino de monges no Ocidente. Este objectivo foi
estabelecido em 1979 com a aquisição de uma casa em ruínas em West Sussex,
posteriormente conhecido como Chithurst Buddhist Monastery ou
\emph{Cittaviveka}.

\enlargethispage{\baselineskip}

Com a inauguração deste mosteiro o Sangha começou a crescer rapidamente,
iniciando também o treino para mulheres como monjas budistas (\emph{siladhara}).
O aumento de pessoas interessadas em viver a vida monástica, ou interessadas em
a apoiar, tornou possível abrir outros mosteiros em Inglaterra e também noutros
países; ajudou também a estabelecer em 1984 um grande centro de treino,
Amarāvatī Buddhist Monastery.

Amarāvatī situa"-se nas imediações da vila de Great Gaddesden perto de Hemel
Hempstead em Hertfordshire. Funciona simultaneamente como mosteiro e centro de
retiros acolhendo aqueles que estão interessados nos ensinamentos do Buddha. Aos
visitantes interessados em viver em comunidade e treinar em termos de
moralidade, meditação e prestação de serviço, é permitida a estadia no mosteiro.

