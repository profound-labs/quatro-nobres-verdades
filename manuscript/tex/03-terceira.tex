\chapterNote{%
  O que é a Nobre Verdade da Cessação do Sofrimento?

  É o desaparecimento do último vestígio e cessação desse mesmo desejo; o
  rejeitar, o abandonar, o deixar e o renunciar do mesmo. Mas onde é que este
  desejo é abandonado e terminado? Onde quer que exista aquilo que parece
  adorável e gratificante, aí é abandonado e terminado.

  Existe esta Nobre Verdade da Cessação do Sofrimento: tal foi a visão, revelação,
  sabedoria, verdadeiro conhecimento e luz que em mim surgiram acerca de coisas
  nunca antes ouvidas.

  Esta Nobre Verdade deve ser penetrada, realizando a Cessação do
  sofrimento\ldots{}

  Esta Nobre Verdade foi penetrada, realizando a Cessação do sofrimento: tal foi a
  visão, revelação, sabedoria, verdadeiro conhecimento e luz que em mim
  surgiram acerca de coisas nunca antes ouvidas.

  \bigskip

  \quoteRef{Saṃyutta Nikāya 56.11}%
}

\chapter{A Terceira Nobre Verdade}

\pagestyle{topbottomcorner}

\section{Existe a Cessação do Sofrimento}

A Terceira Nobre Verdade é composta por três fases: “Existe a cessação do
sofrimento, \emph{dukkha}. O cessar de \emph{dukkha} deve ser realizado. A
cessação de \emph{dukkha} foi realizada”.

Os ensinamentos budistas têm como objectivo fundamental desenvolver uma mente
reflexiva, para que se possam abandonar as ilusões. As Quatro Nobres Verdades
constituem o ensinamento sobre esse abandono, através do olhar atento e da
investigação, observando: “Porque é que é assim? Porque é que é deste modo?”.

É bom ponderar acerca de coisas como: porque é que os monges rapam a cabeça ou o
porquê dos \emph{Buddha"-rūpas}\footnote{%
  \emph{Buddha"-rūpa}: Corpo do Buddha ou Bodhi.} terem a aparência que têm. Nós
contemplamos\ldots{} a mente não está a formar uma opinião sobre estas coisas
serem boas, más, úteis ou desnecessárias. Na verdade a mente está a abrir"-se e a
considerar, “O que é que isto significa? O que é que os monges representam?
Porque é que eles usam a “tigela de oferendas”. Porque é que eles não podem ter
dinheiro? Porque é que eles não podem cultivar os seus próprios alimentos?”.

Observemos como esta forma de vida tem mantido esta tradição, permitindo que a
mesma seja transmitida até aos dias de hoje, desde o seu fundador original, o
Buddha Gautama.

Reflectimos sobre a forma como vemos o sofrimento; como vemos a natureza do
desejo; como reconhecemos que o apego ao desejo é sofrimento, e assim alcançamos
a compreensão interna que permite abandonar o desejo, a realização do
“não"-sofrimento”, a cessação do sofrimento. Estas compreensões só podem surgir
através da reflexão; elas não surgem só porque acreditamos nelas. Não se pode
realizar ou acreditar numa compreensão só pela força da vontade; é através da 
verdadeira contemplação e reflexão destas verdades, que as compreensões surgem. 
Elas só surgem através da abertura e receptividade da mente para com os ensinamentos. 
Fé cega não é certamente, aconselhada ou esperada de ninguém. Em vez disso a mente
deve estar disposta a ficar receptiva, ponderando e considerando.

Este estado mental é muito importante, é o caminho para abandonar o sofrimento.
Não é uma mente com ideias fixas e preconceitos, que pensa que sabe tudo, ou que
só aceita ser verdade o que as outras pessoas dizem. É uma mente que está aberta
para estas Quatro Nobres Verdades e que consegue reflectir acerca de algo que se
pode ver dentro da própria mente.

As pessoas raramente realizam o “não"-sofrimento”, pois este requer uma vontade
muito especial para se poder ponderar e investigar e assim passar para além do
grosseiro e do óbvio. É preciso força de vontade para se observar honestamente
as próprias reacções, ser capaz de reconhecer os apegos e contemplar: “Como é
que é sentir apego?”.

Por exemplo, sentem"-se felizes ou libertos estando apegados ao desejo?
Sentem"-se positivos ou depressivos? Estas são questões a ser investigadas. Se
acharem que estar apegado aos desejos é libertador, então continuem. Apeguem"-se
aos desejos e vejam qual é o resultado.

Na minha prática, tenho observado que o apego para com os meus desejos é
sofrimento. Não tenho qualquer dúvida acerca disso. Consigo ver, quanto
sofrimento na minha vida tem sido causado por apego a coisas materiais, ideias,
atitudes ou medos. Consigo ver todo o tipo de infelicidade desnecessária que
causei a mim próprio, devido a este apego e por não ter compreendido a realidade 
das coisas. Fui criado na América, a terra da liberdade. Ela promete o
direito de ser feliz, mas o que na realidade oferece é o direito de ser apegado
a tudo. A América encoraja a ser o mais feliz possível possuindo coisas. No
entanto, se trabalharem com as Quatro Nobres Verdades, o apego será compreendido
e contemplado, e aí então surge a compreensão do desapego. Isto não é uma posição
intelectual ou uma ordem do cérebro a dizer que não se deve ser apegado; é
apenas uma revelação natural sobre o desapego ou extinção do sofrimento.

\section{A Verdade da Impermanência}

Aqui em Amarāvatī, recitamos o \emph{Sutta Dhammacakkappavattana} na sua forma
tradicional. Quando o Buddha deu este sermão sobre as Quatro Nobres Verdades, só
um dos cinco discípulos que o ouviu o compreendeu verdadeiramente: apenas um
teve a compreensão profunda. Os outros quatro gostaram muito, pensando “Sim
senhor, que ensinamento tão bonito”, mas só Kondañña obteve a perfeita
compreensão acerca daquilo que o Buddha estava a dizer.

Os \emph{devas} também estavam a ouvir o sermão. \emph{Devas} são criaturas
etéreas e celestiais, muito superiores a nós, que não possuem corpos grosseiros
como os nossos mas sim corpos etéreos, e são muito bonitos, gentis e
inteligentes. Mas apesar dos \emph{Devas} se terem deliciado ao ouvir o sermão,
nenhum se iluminou.

Dizem"-nos que eles ficaram muito felizes com a iluminação do Buddha e que
elevaram as suas vozes aos céus quando ouviram o seu ensinamento. Primeiro um nível 
de \emph{devatās} ouviu"-o, depois ecoaram as suas vozes para o próximo nível e 
em pouco tempo todas os devas regozijavam, indo até ao nível mais alto, o reino dos
\emph{Brahmas}. Havia alegria ressoante de que a Roda do Dhamma tinha sido posta
em movimento e estes \emph{devas} e \emph{brahmas} nela regozijavam. Contudo, só
Kondañña, um dos cinco discípulos, se tornou iluminado quando ouviu este sermão.
Mesmo no final do \emph{sutta}, o Buddha chamou"-o de “\emph{Añña} Kondañña”.
“Añña” significa sabedoria profunda, assim “Añña Kondañña”, significa
“Kondañña"-Aquele que Sabe”.

O que é que Kondañña sabia? Qual foi a revelação que o Buddha elogiou no final
do sermão? Foi: “Tudo o que está sujeito a surgir, está sujeito a cessar”. Isto
pode não soar a grande conhecimento, mas o que na realidade implica é um padrão
universal: o que quer que esteja sujeito a surgir está sujeito a cessar; é
impermanente e “não"-eu”\ldots{} assim sendo, não se apeguem, não se iludam com
aquilo que surge e cessa. Não procurem refúgio, naquilo que querem confiar e
respeitar, em nada que surge - pois essas coisas cessarão.

Se quiserem sofrer e desperdiçar a vida, partam à procura das coisas que surgem.
Todas elas vos levarão ao final, à cessação, e não vos tornarão mais sábios por
isso. Continuarão, simplesmente, às voltas neste ciclo, repetindo os mesmos
tristes hábitos e quando morrerem, não terão aprendido nada de importante com a
vida que viveram.

Em vez de só pensarem nisto, contemplem"-no verdadeiramente: “Tudo o que está
sujeito a surgir, está sujeito, a cessar”. Apliquem isto à vida em geral, à
vossa própria experiência e aí irão compreender. Fixem isto: princípio\ldots{}
fim. Contemplem a natureza da vida. Este reino dos sentidos é todo ele feito de
surgir e de cessar, princípio e fim; nesta vida pode"-se alcançar o entendimento
correcto, \emph{sammā diṭṭhi}. Não sei por quanto tempo Kondañña viveu após o
sermão do Buddha, mas nesse momento ele iluminou-se. Nesse mesmo instante ele
obteve entendimento correcto.

Eu gostaria de sublinhar o quão importante é desenvolver este tipo de reflexão.
Em vez de apenas se desenvolver um método para tranquilizar a mente, que
certamente é uma parte da prática, tentem perceber que meditação correcta
envolve dedicação nesta sábia investigação. Observar as coisas profundamente,
envolve um esforço corajoso: não se analisarem a vós próprios ou fazerem juízos
de valor sobre a causa do sofrimento pessoal, mas estarem determinados a seguir
o caminho até obterem um entendimento profundo. Tal entendimento está baseado no
padrão de surgir e cessar. Assim que esta lei for compreendida, tudo é
percebido, adequando"-se a esse padrão.

Isto não é um ensinamento metafísico: “Tudo o que está sujeito a surgir, está
sujeito a cessar”, não se trata da derradeira realidade, a realidade imortal,
\emph{deathless}; mas se souberem de forma profunda e verdadeira, que tudo o que
está sujeito a surgir, está sujeito a cessar, então compreenderão a derradeira
realidade, a verdade imortal. Esta é uma forma engenhosa de alcançar a realização
final. Percebam a diferença: a declaração não é metafísica mas leva"-nos a uma
realização metafísica.

\section{Moralidade e Cessação}

Através da reflexão das Quatro Nobres Verdades, trazemos ao consciente o
problema da existência humana. Observamos esta sensação de alienação e apego
cego à consciência sensorial, o apego para com aquilo que está separado e se
destaca na consciência. Devido à nossa ignorância, apegamo"-nos ao desejo por
prazeres sensoriais, quando nos identificamos com o que é findável ou
transitório e, como tal, insatisfatório. Esse apego torna"-se sofrimento.

Os prazeres sensoriais são todos eles prazeres efémeros. Aquilo que vemos,
ouvimos, tocamos, saboreamos, pensamos ou sentimos é momentâneo, passageiro,
sujeito ao término. Assim, quando nos apegamos aos sentidos ou a sensações
passageiras, apegamo"-nos, por assim dizer, à morte, ao fim.

Se ainda não contemplámos ou compreendemos isto claramente, apegamo"-nos
cegamente à mortalidade na esperança de a evitar por uns tempos. Fingimos
que vamos ser verdadeiramente felizes com as coisas às quais nos apegamos, só
para, eventualmente, nos sentirmos desiludidos, desesperados e desapontados.
Podemos até conseguir tornar"-nos naquilo que desejamos, mas isso também é
momentâneo, findável. No fundo estamos somente a apegar"-nos a outra condição
passageira com um fim certo. Assim, com este desejo de mortalidade podemos vir a
apegar"-nos a ideias de suicídio ou aniquilação, mas o fim propriamente dito é
apenas mais uma condição findável. A que quer que seja que nos apeguemos nestes
três tipos de desejos, estamos a apegar"-nos a algo passageiro e limitado, o que
significa virmos, eventualmente, a sentir desapontamento ou desespero.

A morte da mente é desespero; a depressão é um tipo de experiência de morte da
mente. Tal como o corpo sofre uma morte física, a mente também morre. Estados
mentais e condições mentais morrem; chamamo"-los de desespero, tédio, depressão e
angústia. Se estamos a sentir tédio, desespero, angústia e mágoa, temos a
tendência de procurar qualquer outra condição (findável) que possa surgir para
aliviar essa sensação.

Por exemplo, se alguém se sente desesperado ou entediado, pensa “Preciso de uma
fatia de bolo de chocolate”. E vai comprá"-la! Por uns breves momentos, deixa"-se
envolver no doce, delicioso sabor a chocolate. Nesse momento torna"-se na doçura
e delicioso sabor do chocolate! Mas não consegue suster essa sensação por muito
tempo. Engole o último pedaço de bolo e o que é que resta? Tem de ir procurar
outra forma de alívio. Isto é o devir, “tornar"-se” novamente em algo.

Estamos cegos, aprisionados neste processo de nos tornarmos algo, neste plano
sensorial. Mas conhecendo o desejo, sem julgar a beleza ou feiura do plano
sensorial, chegamos ao ponto de percebermos o desejo tal como ele é. O
conhecimento acontece. Nesse ponto, pondo de lado todos estes desejos em vez de
nos agarrarmos a eles, temos a experiência de \emph{nirodha}, o cessar do
sofrimento. Isto é a Terceira Nobre Verdade que temos de realizar por nós
próprios. Contemplamos a cessação. Dizemos, “Existe cessação”, e sabemos
claramente quando algo cessou.

\section{Permitindo Que as Coisas Surjam}

Antes de se poder deixar as coisas, há que admiti"-las plenamente na consciência.
Na meditação, o nosso objectivo é habilmente permitir que o subconsciente se
manifeste no consciente. Permitimo"-nos ser conscientes de todo o desespero, medo, 
angústia, recalques e irritações. Existe a tendência para as
pessoas se apegarem a grandes ideais mentais, podendo desapontar"-se
verdadeiramente com eles próprios, porque por vezes sentem que não são tão bons
como deveriam ser, ou que não se deveriam zangar - todos aqueles “devemos” e “não
devemos”. Então, criam o desejo de se verem livres das coisas más e este desejo
tem uma característica moralista. Parece certo, verem"-se livres dos maus pensamentos,
raiva e ciúme, porque uma pessoa boa “não devia de ser assim” e, dessa forma
gera"-se a culpa.

Ao reflectirmos sobre isto, tomamos consciência do desejo de nos tornarmos neste
ideal e o desejo de nos libertarmos destas coisas maléficas. Desta forma conseguimos
libertar"-nos e em vez de nos tornarmos na pessoa perfeita, abandonamos esse
desejo. O que fica é a mente pura. Não há qualquer necessidade de sermos a
pessoa perfeita, porque na mente pura é onde as pessoas perfeitas surgem e
cessam.

A cessação é fácil de compreender a nível intelectual, mas realizá"-la pode ser
bastante difícil, pois implica aguentar aquilo que pensamos não conseguir aguentar. 
Por exemplo, quando eu comecei a meditar, pensava que a meditação me tornaria mais
bondoso e mais feliz, estava à espera de sentir estados mentais maravilhosos. No
entanto nunca senti tanto ódio e raiva na minha vida como durante os dois primeiros meses.
Pensei “isto é terrível, a meditação tornou"-me pior”, mas então observei
porque razão surgiu tanto ódio e tanta aversão e compreendi que grande parte da
minha vida tinha sido uma tentativa de fugir a tudo isso. Era um leitor
compulsivo e para onde quer que fosse tinha de levar livros comigo. Sempre que o
medo ou a aversão surgiam, pegava num livro para ler, ou fumava um cigarro, ou
comia um “\emph{snack}”. A imagem que tinha de mim próprio era de uma pessoa
bondosa que não odiava os outros, e assim qualquer indício de aversão ou ódio eram
reprimidos.

Esta foi a razão porque durante os primeiros meses como monge, estava tão
desesperado para que tudo isto desaparecesse. Tentava procurar algo para me distrair,
porque com a meditação tinha começado a relembrar todas as coisas que
deliberadamente tentei esquecer. Memórias de infância e adolescência surgiam
constantemente na minha mente, e nesse ponto, a raiva e o ódio tornaram"-se tão
conscientes, que pareciam ser maiores que eu. Mas algo em mim começou a
reconhecer que tinha de aguentar tudo isto e assim o fiz. Todo o ódio e raiva
que tinham sido suprimidos durante trinta anos de vida vieram em força, mas
através da meditação extinguiram"-se e desapareceram. Foi um processo de
purificação.

Para permitirmos que este processo de cessação se dê, temos de estar dispostos a
sofrer. É por essa razão que eu reforço a importância de se ser paciente. Temos
de abrir as nossas mentes ao sofrimento porque é no acolher do sofrimento que o
mesmo cessa. Quando sentimos que estamos a sofrer, física ou mentalmente, temos
de ir ao encontro desse sofrimento. Abrimo"-nos a ele completamente, damos"-lhe
as boas vindas e concentramo"-nos nele, permitindo"-o ser aquilo que é. Isso
significa que temos de ser pacientes e aguentar as condições menos agradáveis,
em vez de fugirmos, temos de aguentar o tédio, o desespero, a dúvida e o medo para
podermos compreender que os mesmos cessam.

Enquanto não permitirmos que as coisas cessem, continuamos a criar novo
\emph{kamma} que só ajuda a fortalecer os nossos hábitos. Quando algo surge,
agarramo"-nos e proliferamos sobre isso, o que torna tudo ainda mais complicado 
e assim, repetimos e tornamos a repetir o mesmo padrão durante a nossa vida - não podemos
continuar a seguir os nossos desejos e medos e esperar que algum dia vamos realizar a paz.
Observamos o medo e o desejo para que estes deixem de nos iludir: temos de
conhecer aquilo que nos ilude antes que nos possamos libertar. Desejo e medo
devem ser reconhecidos como impermanentes, insatisfatórios e como “não"-eu”. Eles
são observados e compreendidos para que o sofrimento possa cessar.

É importante aqui diferenciar entre cessação, o fim natural de qualquer condição
que tenha surgido, e aniquilação, o desejo (que surge na mente) de nos vermos
livres de algo. Daí a cessação não ser desejo! Não é algo que criamos na mente,
mas sim o fim daquilo que começou, a morte daquilo que nasceu. Daí a cessação não
ser um 'eu', não se manifesta a partir do ponto em que “Eu tenho de me ver livre
destas coisas”, mas somente quando permitimos que aquilo que surgiu cesse. Para
conseguir isso, o desejo tem de ser abandonado - deixá"-lo ir. Isto não significa
rejeitar ou deitar fora, mas sim largá"-lo.

Então, quando ele cessar temos a experiência de \emph{nirodha}, cessação, vazio,
desapego. \emph{Nirodha} é outra palavra para Nibbāna. Quando se abre mão
de algo, permitindo que cesse, tudo o que resta é paz.

Podemos viver essa paz através da própria meditação, quando na nossa mente
deixarmos o desejo terminar: aquilo que sobeja é muito sereno. Isso é paz
verdadeira, \emph{deathless} (sem"-morte). Quando conhecemos isso, tal como é
verdadeiramente, realizamos \emph{nirodha sacca}, a Verdade da Cessação, na qual
deixa de existir o 'eu', mas, ainda existe vigilância e claridade. O verdadeiro
significado da felicidade é essa serenidade, consciência transcendente.

Se não permitirmos a cessação, então a tendência é para funcionarmos a partir das
suposições que fazemos acerca de nós mesmos, sem sequer sabermos o que estamos a
fazer. Às vezes, só quando começamos a meditar é que nos apercebemos o quanto o
medo e a falta de confiança que sentimos, na nossa vida, provêm das experiências
da nossa infância. Lembro"-me de quando era miúdo ter um grande amigo que um dia
se voltou contra mim e me rejeitou. Durante meses andei desesperado e isto
deixou uma marca indelével na minha mente. Então, realizei através da meditação,
o quanto um pequeno incidente como esse, veio a afectar as minhas futuras
relações com os outros. Sempre tive um medo tremendo da rejeição. Nunca tinha
pensado nisso até essa memória continuar a surgir no meu consciente durante a
meditação. A mente racional sabe que é ridículo continuar a pensar nas tragédias
de infância mas, se as mesmas continuam a surgir no consciente, quando se chega
à meia"-idade, talvez estejam a querer dizer algo sobre os conceitos que formamos
quando crianças.

Quando na meditação surgem memórias ou medos obsessivos, ao invés de se sentirem
frustrados ou irritados, vejam"-nos como algo a ser aceite no consciente para
desta forma, os poderem libertar. Podem organizar a vida de modo a nunca terem
de olhar para estas coisas; assim as possibilidades de surgirem tornam"-se
mínimas. Podem dedicar"-se a muitas causas importantes e manterem"-se sempre
ocupados; assim, estas ansiedades e medos sem nome nunca se tornarão em algo
consciente. Mas o que é que acontece quando pararem e deixarem de controlar? O
desejo ou a obsessão alteram"-se, movendo"-se na direcção da cessação. Eles findam 
e então adquire"-se a sabedoria de que existe a cessação do desejo. Assim, em 
conclusão, a terceira fase da Terceira Nobre Verdade é: a cessação foi realizada.

\section{Realização}

Isto é para ser realizado. Disse o Buddha enfaticamente: «Isto é uma Verdade a
ser realizada aqui e agora». Não precisamos de esperar até morrer para descobrir
se tudo isto é verdade - este ensinamento é destinado a todos os seres humanos.
Cabe a cada um de nós realizá"-lo. Eu posso explicar e encorajar"-vos a praticar
mas não posso fazer com que o realizem!

Não pensem nele como algo remoto, como algo para além das vossas possibilidades. Quando
falamos sobre Dhamma ou Verdade, dizemos que está aqui e agora, algo que podemos
observar por nós próprios. Podemos inclinar"-nos para a Verdade. Podemos prestar
atenção à forma como as coisas são, aqui e agora, neste momento e neste lugar.
Isso é estar consciente, estar alerta e focar a atenção na forma como as coisas
são. Com esta consciência, investigamos o sentido do eu, esta sensação de mim
e daquilo que é meu: o meu corpo, os meus sentimentos, as minhas memórias, os
meus pensamentos, as minhas opiniões, a minha casa, o meu carro e por aí fora.

A minha tendência era ser depreciativo de mim próprio, por exemplo com o
pensamento “Eu sou Sumedho”, pensava em termos negativos acerca de mim mesmo “Eu
não presto”. Mas atenção, onde é que isso surge e onde é que cessa?\ldots{} ou,
“Eu sou muito melhor que vós, sou muito mais avançado. Há bastante tempo que
vivo a Vida Santa, por isso, devo ser melhor que qualquer um de vós!”. De onde é
que isto surge e onde é que cessa?

Quando houver arrogância, presunção ou depreciação própria, o que quer que seja,
examine"-se e escute"-se o que vai dentro de cada um “Eu sou\ldots{}”. Estejam
atentos e conscientes do espaço antes de pensarem, depois pensem no espaço e
reparem no que se segue. Mantenham a vossa atenção no vazio no final do
pensamento e, vejam por quanto tempo conseguem manter a atenção. Vejam se
conseguem ouvir um tipo de som na mente, o som do silêncio, o som primordial.
Quando concentrarem a atenção nisso podem reflectir: “Existe alguma sensação de
eu?” e verão que quando estão realmente vazios, quando só existe clareza,
vigilância e consciência, não existe nenhum eu. Não existe a sensação de mim ou meu, 
e assim vou para esse estado de vazio e contemplo o Dhamma. Penso “Isto é como
deve ser. Este corpo aqui presente é desta forma”. Eu posso dar"-lhe um
nome ou não, mas neste preciso momento, é simplesmente assim; não é Sumedho!

No vazio não existe nenhum monge budista. Monge budista é simplesmente uma
convenção apropriada ao espaço e ao tempo. Quando as pessoas vos elogiam
dizendo, “Que maravilhoso”, podem interpretá"-lo como alguém a oferecer um elogio
sem terem necessariamente de o tomar pessoalmente. Pois sabem que não existe
nenhum monge budista; mas só o que~é. É claramente assim. Se eu quiser que
Amarāvatī seja um grande sucesso e se isso acontecer, eu fico feliz. Mas se
falhar, se ninguém se mostrar interessado e não pudermos pagar a conta da
electricidade e, se tudo se desmoronar - falhanço! Na realidade não existe
nenhum Amarāvatī. A ideia da pessoa que é um monge budista ou o lugar chamado
Amarāvatī são apenas convenções e não a derradeira realidade. Neste preciso
momento é simplesmente assim, tal como deveria ser. Quando vemos tal lugar como
realmente é, não o carregamos aos ombros, pois percebemos que não existe pessoa
alguma para se envolver nesse processo. Quer ele seja bem sucedido ou falhe, deixa de ter importância.

No vazio, as coisas são apenas aquilo que são. Quando estamos desta forma conscientes, 
não significa que somos indiferentes ao sucesso ou ao fracasso e que
não nos preocupamos em fazer coisa alguma. Podemos aplicar"-nos, sabemos o
que podemos, sabemos o que deve ser feito e podemos fazê"-lo da forma correcta.
Aí tudo se torna Dhamma, tal como é. Fazemos as coisas porque é aquilo que é
correcto fazer neste momento, e neste lugar, em vez de ser por ambição pessoal
ou medo de fracasso.

O caminho para a cessação do sofrimento é o caminho da perfeição. Perfeição pode
ser uma palavra muito intimidante porque nos sentimos muito imperfeitos. Como
personalidades questionamo"-nos como podemos sequer atrever"-nos a considerar a
possibilidade de sermos perfeitos. A perfeição humana é algo acerca do qual
ninguém fala; não parece ser sequer possível pensar na perfeição em termos de se
ser humano. Mas um \emph{Arahant} é nada mais nada menos do que um ser humano que 
aperfeiçoou a própria vida. Alguém que aprendeu tudo o que há para aprender através 
da lei básica “Tudo o que está sujeito a surgir, está sujeito a cessar”. Um
\emph{Arahant} não necessita de saber tudo acerca de tudo, só é necessário saber
e compreender plenamente esta lei.

Usamos a sabedoria do Buddha para contemplar o Dhamma, a forma como as coisas
são. Tomamos como refúgio o Sangha, naquilo que faz bem e no que se abstém de
fazer mal. Sangha não é um grupo de personalidades individuais ou de diferentes
carácteres, é uma comunidade. A noção de ser um indivíduo ou um homem ou uma
mulher deixa de ser algo importante para nós. Esta noção de Sangha é realizada
como refúgio. Ainda que as manifestações sejam todas individuais, a nossa
realização é a mesma -- existe uma unidade. Com este despertar, estado de alerta e
desapego, realizamos a cessação e permanecemos no vazio no qual todos nos
fundimos. Assim, não existe nenhum indivíduo, as pessoas podem surgir e cessar
no vazio, mas não existe nenhuma pessoa, somente claridade, plenitude,
serenidade e pureza.

