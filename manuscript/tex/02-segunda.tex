\chapterNote{%
  O que é a Nobre Verdade da Origem do Sofrimento?

  É o desejo que renova a
  existência e é acompanhado pela cobiça e prazer, cobiçando isto e aquilo: desejo pelos prazeres sensoriais, desejo por ser, desejo por não ser.
  Mas onde nasce e floresce este desejo? Onde quer que exista algo adorável e
  gratificante, aí ele nasce e floresce.

  Existe esta Nobre Verdade da Origem do Sofrimento: tal foi a visão, revelação,
  sabedoria, verdadeiro conhecimento e luz realizadas acerca de coisas nunca
  antes ouvidas.

  Esta Nobre Verdade deve ser penetrada abandonando a origem do
  sofrimento\ldots{}

  Esta Nobre Verdade foi penetrada tendo abandonado a origem do sofrimento: tal
  foi a visão, revelação, sabedoria, verdadeiro conhecimento e luz realizadas
  acerca de coisas nunca antes ouvidas.

  \quoteRef{Saṃyutta Nikāya 56.11}%
}

\chapter{A Segunda Nobre Verdade}

\section{Existe a Origem do Sofrimento}

A Segunda Nobre Verdade é composta por três fases: “Existe a origem do
sofrimento que é o apego ao desejo. O desejo deve ser abandonado. O desejo foi
abandonado”.

A Segunda Nobre Verdade diz-nos que existe uma origem para o sofrimento e que
essa origem se encontra nos três tipos de desejo: desejo de prazeres sensoriais
(\emph{kāma taṇhā}), desejo de ser (\emph{bhava taṇhā}) e o desejo de não ser
(\emph{vibhava taṇhā}). Esta é a declaração da Segunda Nobre Verdade, a tese,
\emph{pariyatti}. Isto é o que se contempla: a origem do sofrimento encontra-se
no apego ao desejo.

\section{Três Tipos de Desejo}

Desejo ou \emph{taṇhā} em Pāli é algo importante a ser compreendido. O que é o
desejo? \emph{Kāma taṇhā} é muito fácil de perceber. Este tipo de desejo é
querer prazeres sensoriais através do corpo e procurar sempre coisas que excitem
e agradem aos sentidos. Podem efectivamente contemplar: o que é que se passa
quando surge o desejo de prazer? Por exemplo, quando estão a comer, se têm fome
e a comida é deliciosa podem estar conscientes de quererem mais uma “garfada”.
Observem essa sensação de saborearem algo agradável e reparem como querem mais.
Não acreditem nisto só por acreditar, experimentem pessoalmente. Não pensem que
sabem o que isto é, só porque foi sempre assim, façam a experiência quando
comerem outra vez, saboreiem algo delicioso e observem o que se passa a seguir:
o surgir do desejo de querer mais. Isso é \emph{kāma taṇhā}.

Também contemplamos a sensação de querermos ser algo. Se houver ignorância,
quando não estamos a procurar algo delicioso para comer ou alguma boa música
para ouvir, podemos então, encontrarmo-nos presos num mundo de ambição e
conquista, o desejo de vir a ser. Somos apanhados nesse movimento, nessa luta
para nos tornarmos felizes; procuramos formas de enriquecer ou tentamos tornar a
nossa vida em algo importante, ao empenharmo-nos em pôr o mundo em ordem. Assim,
apercebam-se desta sensação de querer ser algo mais do que aquilo que na
realidade são neste momento.

Escutem o \emph{bhava taṇhā} na vossa vida: “Quero praticar meditação para poder
ser livre da minha dor. Quero tornar-me iluminado. Quero ser um monge ou uma
monja. Quero atingir a iluminação como pessoa leiga. Quero ter uma mulher e
filhos e uma boa profissão. Quero gozar o mundo dos sentidos sem ter que abdicar
de nada e tornar-me num \emph{Arahant} (Ser Nobre)”.

Quando nos desiludimos por termos querido ser algo, então acabamos por desejar
vermo-nos livres de certas coisas. Então contemplamos \emph{vibhava taṇhā}, o
desejo de nos libertarmos: “Quero ver-me livre do meu sofrimento. Quero ver-me
livre da minha raiva. Tenho esta irritação e quero ver-me livre disto. Quero
libertar-me da inveja, do medo e da ansiedade”. Atenta nisto como uma reflexão
sobre o desejo de não ser \emph{vibhava taṇhā}. Na realidade, estamos a
contemplar aquilo que dentro de nós se quer ver livre das coisas; não estamos a
tentar ver-nos livres do \emph{vibhava taṇhā}. Não estamos a tomar a posição de
estar contra o desejo de nos querermos ver livres das coisas, nem estamos a
encorajar esse desejo. Em vez disso, estamos a reflectir “É desta forma; é assim
que nos sentimos quando nos queremos ver livres de algo; eu tenho de conquistar
a minha ira; tenho de matar o Diabo e ver-me livre do meu egoísmo, aí eu
serei\ldots{} ” Podemos observar por esta corrente de pensamentos que, querer ser e
querer vermo-nos livres estão bastante interligados.

Porém, convém ter em mente que estas três categorias de \emph{kāma taṇhā},
\emph{bhava taṇhā} e \emph{vibhava taṇhā} são apenas métodos convenientes para
contemplarmos o desejo. Elas não são formas de desejo totalmente diferentes mas
sim diferentes aspectos do mesmo.

A segunda revelação da Segunda Nobre Verdade é: “O desejo deve ser abandonado”.
É assim que o abandonar surge na nossa prática. Surge a revelação de que o
desejo deve ser abandonado, mas essa revelação não é um desejo de se querer ver
livre de nada. Se não se for suficientemente sensato e se não houver reflexão,
tem-se a tendência a seguir o “eu quero ver-me livre de\ldots{}, quero
libertar-me de todos os meus desejos”, porém, isto é apenas outro desejo. No
entanto, pode-se reflectir nele; pode-se observar o desejo de se querer libertar
do desejo de querer ser ou do desejo de prazeres sensoriais. Compreendendo estes
três tipos de desejo, pode-se deixá-los.

A Segunda Nobre Verdade não vos pede que pensem, “Eu tenho muitos desejos
sensoriais”, ou “Sou mesmo ambicioso. Sou todo \emph{bhava taṇhā}: mais, mais,
mais!” ou, “Sou um verdadeiro niilista. Só quero desaparecer. Sou um verdadeiro
fanático do \emph{vibhava taṇhā}”. A Segunda Nobre Verdade não é nada disso. Não
se trata de forma alguma de identificação com os desejos, mas sim, do
reconhecimento desses desejos.

Costumava perder bastante tempo a observar o quanto da minha prática era desejo
de ser algo. Por exemplo, quanto das boas intenções da minha prática de
meditação como monge eram para que gostassem de mim, quanto do meu
relacionamento com os outros monges ou monjas, ou com as pessoas leigas, tinha a
ver com o desejo de ser apreciado e respeitado. Isto é \emph{bhava taṇhā},
desejo de ter elogios e sucesso. Como monges, temos este \emph{bhava taṇhā}; o
querer que as pessoas percebam tudo e que apreciem o Dhamma, até estes subtis,
quase nobres desejos são \emph{bhava taṇhā}.

Depois temos \emph{vibhava taṇhā} na vida espiritual, que pode parecer muito
virtuoso: “Eu quero ver-me livre, aniquilar e exterminar estas contaminações da
mente”. De facto, dava comigo a pensar, “Eu quero ver-me livre dos desejos.
Quero ver-me livre da raiva. Nunca mais quero ter medo ou inveja. Eu quero ser
corajoso. Eu quero ter alegria e felicidade no meu coração”.

Esta prática do Dhamma, não é para nos odiarmos por termos tais pensamentos mas,
para observar claramente que estes são condicionados pela mente. Eles são
impermanentes. O desejo não é aquilo que somos, mas a forma como reagimos devido
à nossa ignorância, quando ainda não compreendemos estas Quatro Nobre Verdades
nos seus três aspectos. Temos tendência para reagir desta maneira e estas são
reacções normais devido à ignorância.

Contudo, não necessitamos de continuar a sofrer, não somos necessariamente
vítimas desesperadas do desejo. Podemos deixar o desejo ser da forma que é, e
assim, começarmos a libertar-nos dele. O desejo só nos ilude e tem poder sobre
nós, enquanto o agarramos, acreditamos e a ele reagimos.

\section{Apego é Sofrimento}

Normalmente comparamos sofrimento com sentimento, mas sentimento não é
sofrimento. É o apego ao desejo que é sofrimento. O desejo não causa sofrimento,
a causa do sofrimento é o apego ao desejo. Esta declaração serve para reflexão e
contemplação em termos da experiência de cada um.

Têm mesmo que investigar e conhecer verdadeiramente o desejo. Têm de saber o que
é, e o que não é natural e necessário para sobreviver. Podemo-nos tornar muito
idealistas e pensar que até a necessidade de alimento é um tipo de desejo que
não deveríamos ter, podemo-nos tornar bastante ridículos por causa disso. Mas o
Buddha não era um idealista nem um moralista, ele não tentou condenar fosse o
que fosse, ele tentou despertar-nos para a verdade, para que pudéssemos ver as
coisas claramente.

Quando essa clareza e visão correcta estiverem presentes deixa de haver
sofrimento. Podem continuar a sentir fome, podem continuar a precisar de
alimentos sem que isto se torne um desejo. Os alimentos são uma necessidade
natural do corpo, o corpo não é o eu, ele necessita de alimentos ou então,
torna-se fraco e morre; essa é a natureza do corpo, não há nada de errado com
isso. Se nos tornarmos todos moralistas e acreditarmos que somos o nosso corpo,
que essa fome é o nosso problema, e que nem devemos comer, tal não revela
sabedoria, é simplesmente idiotice.

Quando realmente virem a origem do sofrimento, compreenderão que o problema é o
apego ao desejo e não o desejo em si. Apegarem-se significa deixarem-se iludir
pelo desejo e, então, começa-se a pensar em função do “eu” e é “meu”; “Estes
desejos são quem eu sou e decerto algo está errado comigo por os sentir”; ou “Eu
não gosto de ser como sou. Tenho de me tornar em algo diferente”; ou então
“Tenho de me livrar disto antes de me poder tornar naquilo que quero ser”. Tudo
isto é desejo. Escutem atentamente tudo isto sem comentarem o que é bom ou o que
é mau, mas meramente reconhecendo-o pelo que é.

\section{Desapego}

Se contemplarmos e escutarmos os desejos, deixamos de estar apegados a eles,
estamos somente a deixá-los ser como são. Então chegamos à conclusão de que a
origem do sofrimento, do desejo, pode ser posto de lado e abandonado.

Como é que se largam as coisas? Deixando-as tal como são; não significa que as
aniquilamos ou as deitamos fora. É como que pô-las de lado e deixá-las ficar.
Através da prática do desapego apercebemo-nos que existe a origem do sofrimento,
que é o apego ao desejo e, compreendemos que devemos largar estes três tipos de
desejo. Então, apercebemo-nos que deixámos estes desejos e que deixou de haver
qualquer apego a eles.

Quando estiverem apegados, lembrem-se que desapego não é “verem-se livres de” ou
“deitar fora”. Se eu estiver agarrado a este relógio e me disserem “Deixa-o!”,
tal não significa “deitá-lo fora”. Posso pensar que tenho de o deitar fora
porque estou apegado a ele, porém, seria apenas o desejo de me ver livre dele.
Geralmente pensamos que, ficarmos livres de um objecto é uma forma de nos vermos
livres do apego. Mas se eu conseguir contemplar o apego a este relógio,
compreendo que não existe qualquer razão para me ver livre dele, é um bom
relógio, está sempre certo e nem sequer é muito pesado. O relógio não é o
problema. O problema é apegar-me a ele. Então o que é que eu faço? Largo-o,
ponho-o de parte, coloco-o cuidadosamente de lado, sem qualquer tipo de aversão.
Depois posso voltar a pegar-lhe, ver que horas são e pô-lo de parte quando não
for necessário.

Pode-se aplicar esta sabedoria do desapego aos desejos sensoriais. Por exemplo,
uma pessoa que queira muito divertir-se. Como é que poria de parte esse desejo
sem qualquer aversão? Simplesmente reconhecendo o desejo sem fazer juízos de
valor. Pode-se contemplar o querer ver-se livre dele - porque se sente culpado
ao ter um desejo tão tolo -- basta simplesmente pô-lo de lado. Então, quando se
vê como ele realmente é, reconhecendo que é somente um desejo, deixa-se de estar
apegado a ele.

Assim, o caminho é trabalharem sempre com os momentos da vida diária. Quando se
sentirem deprimidos e negativos, no preciso momento em que se recusam a entregar
a essa sensação, já estão a viver uma experiência iluminada. Quando vêem isso já
não têm de se afundar no mar da depressão e do desespero. Podemos parar e
perceber que não devemos dar azo a um segundo pensamento.

Têm de aprender isto pela própria prática e experiência, para que possam saber
por vós próprios como se libertarem da origem do sofrimento. Será que podem
libertar-se do desejo por simplesmente quererem desapegar-se dele? O que é que
está realmente a ser abandonado neste momento? Têm de contemplar a experiência
do abandonar e verdadeiramente investigar e examinar, até que a realização
surja. Continuem até que o verdadeiro saber chegue: “Ah, o desapego! Sim, agora
compreendo! O desejo foi abandonado!”. Isto não significa que se vá abandonar o
desejo para sempre, contudo, nesse breve momento, realmente foi abandonado e foi
feito conscientemente. Então surge a realização. É a isto que
chamamos sabedoria plena. Em Pāli, chamamo-lo de \emph{ñāṇadassana} ou
compreensão profunda.

Eu tive a minha primeira revelação no que respeita ao desapego, no meu primeiro
ano de meditação. Eu compreendi intelectualmente que temos de abandonar tudo e
depois pensei: “Como é que se abandona?” Parecia impossível abandonar fosse o
que fosse. Continuei a contemplar: “Como é que se abandona?” depois dizia,
“Abandonas, abandonando”. “Bem, então abandona!” Depois dizia: “Mas será que já
abandonei?” e “Como é que podes abandonar?” “Bem, simplesmente abandonando!” E
assim continuei tornando-me cada vez mais frustrado. Mas, eventualmente,
tornou-se óbvio o que estava a acontecer. Se tentarem analisar como abandonar em
pormenor, torna-se tudo muito mais complicado. Já não se tratava de algo que
pudesse ser expresso por palavras, mas algo que simplesmente fazia. E assim por
um momento eu abandonei tudo, assim simplesmente.

No que respeita a problemas pessoais e obsessões, o método para o desapego é o
mesmo. Não se trata de analisar exaustivamente e tornar o problema ainda maior
mas, de praticar esse estado de deixar as coisas em paz, de largá-las. De
início, põe-se de parte mas depois torna-se a pegar porque o hábito do apego é
muito forte. Mas pelo menos fica-se com a ideia. Mesmo após ter tido essa
revelação acerca do desapego, eu era capaz de abandonar por uns momentos mas
depois voltava a apegar-me, com o pensamento: “Não consigo fazê-lo, tenho tantos
maus hábitos!”.

Mas não confiem nesse tipo de constante crítica depreciativa dentro de vós. Não
é digno de confiança. É simplesmente uma questão de praticar o desapego. Quanto
mais vezes observarem como se faz, mais facilmente conseguirão manter esse
estado de desapego.

\section{Realização}

É importante saberem quando se deu o desapego do desejo, quando deixaram de
fazer juízos de valor ou quando deixaram de tentar livrar-se deles: quando
reconheceram que esta é a forma como as coisas são. Quando se está
verdadeiramente calmo e em paz, percebe-se que não existe apego a nada. Deixa-se
de estar prisioneiro, quer tentando ter algo, quer libertando-se de algo.
Bem-estar é simplesmente conhecer as coisas como elas realmente são, sem sentir
a necessidade de fazer sobre elas qualquer juízo de valor.

Estamos constantemente a dizer, “Isto não devia de ser assim!”, “Eu não devia de
ser como sou!” e “Tu não devias de ser assim e tu não devias de fazer isso!” e
por aí fora\ldots{} Tenho a certeza que vos poderia dizer o que deveriam ser e
vocês conseguiriam dizer-me o que eu deveria ser. Nós deveríamos ser gentis,
carinhosos, generosos, trabalhadores, diligentes, corajosos, compassivos e com
bom coração. Eu não tenho sequer que vos conhecer para vos dizer isto! Mas, para
vos conhecer realmente, eu teria de me abrir convosco, em vez de começar a
partir de um idealismo sobre o que uma mulher ou um homem deve ser, o que um
budista deve ser ou, o que um cristão deve ser.

O nosso sofrimento provém do apego que temos para com os nossos ideais e das
complexidades que criamos sobre a forma como as coisas são. Nós nunca somos o
que deveríamos ser de acordo com os nossos ideais mais altos. A vida, os outros,
o país em que estamos, o mundo em que vivemos, as coisas nunca parecem ser
aquilo que desejariamos que fossem. Tornamo--nos muito críticos de tudo e de nós
mesmos: “Sei que deveria ser mais paciente, mas eu NÃO consigo ser
paciente!”\ldots{} Ouçam bem todos estes “deveria” e “não deveria”, os desejos
de querer o que é agradável, querer ser ou querer ver-se livre daquilo que é
feio ou do que é doloroso. É como ouvir alguém a falar do outro lado da cerca
dizendo: “Eu quero isto e eu não gosto daquilo. Deveria de ser assim e não
assado!”. Disponibilizem-se de tempo para ouvir a mente contestadora; tragam-na
para o consciente.

Eu costumava fazer muito isto quando me sentia insatisfeito ou crítico. Fechava
os olhos e começava a pensar, “Eu não gosto disto e não quero aquilo”, “Aquela
pessoa não devia de ser assim” e “O mundo não deveria de ser assado”. Continuava
a ouvir este tipo de demónio crítico que falava, falava, criticava-me a mim, aos
outros e ao mundo. E então pensava, “Quero felicidade e conforto. Quero
sentir-me seguro. Quero ser amado!”. Eu pensava nestas coisas deliberadamente e
ouvia-as para assim puder conhecê-las apenas como condições que nascem na mente.
Assim sendo, tragam-nas à tona da vossa mente, despertem todas as esperanças,
desejos e críticas; tragam-nas ao consciente e dessa forma conhecerão o desejo e
poderão pô-lo de lado.

Quanto mais contemplamos e investigamos o apego, mais revelações surgem; o
desejo deve ser abandonado. Deste modo, através da própria prática e compreensão
do que realmente significa abandonar, obtemos a terceira revelação da Segunda
Nobre Verdade: “O desejo foi abandonado”. Efectivamente conhecemos o desapego.
Não é um desapego teórico mas uma revelação directa. Agora sabem que o desapego
foi concretizado. Isto é tudo o que a prática é.

