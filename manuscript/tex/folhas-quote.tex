\chapter{Uma mão cheia de folhas}

A certa altura, estando O Iluminado a viver em Kosambi
numa floresta de Simsapās, pegou numas quantas folhas e
perguntou aos monges, «O que é que vocês pensam disto,
monges? O que é mais numeroso, as poucas folhas que tenho
na mão ou aquelas nas árvores desta floresta?».
«As folhas que O Iluminado tem na mão são poucas,
Senhor; as da floresta são bastante mais numerosas».
«Assim também, monges, as coisas que eu aprendi por
conhecimento directo são bastante numerosas; as coisas que
eu vos ensinei são poucas.
E porque é que eu não ensinei todas? Porque elas não
trazem qualquer benefício, nem desenvolvimento na Vida
Santa, porque não conduzem ao fim da ilusão, ao despojamento, à cessação, ao acalmar, ao conhecimento directo, à iluminação, à libertação. Por essa razão não as
ensinei.
E o que é que eu vos ensinei? Existe o sofrimento;
existe a origem do sofrimento; existe o cessar do sofrimento;
existe o caminho que conduz à cessação do sofrimento. Isto
foi o que vos ensinei. E porque é que eu ensinei isto?
Porque traz benefício e desenvolvimento na Vida Santa,
porque conduz ao fim da ilusão, ao despojamento, à cessação, ao acalmar, ao conhecimento directo, à iluminação,
à libertação. Assim sendo monges, que esta seja a vossa
tarefa: Existe o sofrimento; existe a origem do sofrimento;
existe o cessar do sofrimento; existe o caminho que leva à
cessação do sofrimento».

Samyutta Nikāya, LVI, 31
