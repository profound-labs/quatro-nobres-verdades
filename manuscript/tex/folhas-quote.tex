\newpage\mbox{}
\thispagestyle{empty}
\newpage\mbox{}
\thispagestyle{empty}

\makeatletter
\AddToShipoutPictureFG*{%
  \put(\LenToUnit{-3mm},\LenToUnit{-3mm}){%
  %
  \begin{tikzpicture}
    \node (bg) [
      minimum width = \BOOK@paperWidth + 6mm,
      minimum height = \BOOK@paperHeight + 6mm,
    ] {};
    %
    \node (title) [
      minimum width=\BOOK@paperWidth + 6mm,
      below right=18mm and 0pt of bg.north west,
    ] {%
      \begin{minipage}{\BOOK@paperWidth}%
        \centering
        {%
          \chapterTitleFont\chapterTitleSize\color{chaptertitle}%
          Uma Mão Cheia de Folhas%
        }%
      \end{minipage}%
    };%
    %
    \node (quote) [
      minimum width=\BOOK@paperWidth + 6mm,
      below right=8mm and 0pt of bg.mid,
      anchor=mid,
    ] {%
      \begin{minipage}{90mm}%
        \setlength{\parindent}{0pt}%
        \setlength{\parskip}{5pt}%
\vspace*{0.5\onelineskip}%
\fontsize{10}{13}\selectfont%
\color{chapternote}%

A certa altura, estando O Iluminado a viver em Kosambi numa floresta de
Siṃsapās, pegou numa mão cheia de folhas e perguntou aos monges,

«Monges, o que pensam disto? O que é mais numeroso, as poucas folhas
que tenho na mão ou aquelas nas árvores desta floresta?».

«As folhas que O Iluminado tem na mão são poucas, Senhor; as da floresta são
bastante mais numerosas».

«Assim também, monges, as coisas que eu aprendi por conhecimento directo são
bastante numerosas; as coisas que eu vos ensinei são poucas.

E porque é que eu não ensinei todas? Porque elas não trazem qualquer benefício,
nem desenvolvimento na Vida Santa, porque não conduzem ao fim da ilusão, ao
despojamento, à cessação, ao acalmar, ao conhecimento directo, à iluminação, à
libertação. Por essa razão não as ensinei.

E o que é que eu vos ensinei? Existe o sofrimento; existe a origem do
sofrimento; existe o cessar do sofrimento; existe o caminho que conduz à
cessação do sofrimento. Isto foi o que vos ensinei.

E porque é que eu ensinei isto? Porque traz benefício e desenvolvimento na Vida
Santa, porque conduz ao fim da ilusão, ao despojamento, à cessação, ao acalmar,
ao conhecimento directo, à iluminação, à libertação.

Assim sendo monges, que esta seja a vossa tarefa: Existe o sofrimento; existe a
origem do sofrimento; existe o cessar do sofrimento; existe o caminho que leva à
cessação do sofrimento».

\vspace*{5pt}

\quoteRef{Saṃyutta Nikāya 56.31}

\vspace*{0.5\onelineskip}%

      \end{minipage}%
    };%
    %
    \node (topbox) [
      minimum width=\BOOK@paperWidth + 6mm,
      minimum height=10pt,
      above=1mm of quote.north,
      anchor=south,
    ] {};
    %
    \draw [line width=2pt, draw=black!40]
      (topbox.north west) -- (topbox.north east);
    %
    \draw [line width=1pt, draw=black!40, dash pattern=on 1pt off 5pt]
      (topbox.south west) -- (topbox.south east);
    %
    \node (bottombox) [
      minimum width=\BOOK@paperWidth + 8mm,
      minimum height=10pt,
      below=1mm of quote.south,
      anchor=north,
    ] {};
    %
    \draw [line width=1pt, draw=black!40, dash pattern=on 1pt off 5pt]
      (bottombox.north west) -- (bottombox.north east);
    %
    \draw [line width=2pt, draw=black!40]
      (bottombox.south west) -- (bottombox.south east);
    %
  \end{tikzpicture}%
}}%
\makeatother

