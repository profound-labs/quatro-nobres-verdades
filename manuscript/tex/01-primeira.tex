\chapterNote{%
  O que é a Nobre Verdade do Sofrimento?

  Nascimento é sofrimento, envelhecimento
  é sofrimento e morte é sofrimento. Estarmos separados daquilo que gostamos é
  sofrimento, não obter aquilo que queremos é sofrimento: em resumo, os cinco
  agregados influenciados pelo apego são sofrimento.

  Existe esta Nobre Verdade do Sofrimento: tal foi a visão, revelação,
  sabedoria, verdadeiro conhecimento e luz que em mim surgiram,
  acerca de coisas nunca antes ouvidas.

  Esta Nobre Verdade deve ser penetrada através da completa compreensão do
  sofrimento: tal foi a visão, revelação, sabedoria, verdadeiro conhecimento e
  luz que em mim surgiram, acerca de coisas nunca antes ouvidas.

  Esta Nobre Verdade foi penetrada através da completa compreensão do
  sofrimento: tal foi a visão, revelação, sabedoria, verdadeiro conhecimento e
  luz que em mim surgiram, acerca de coisas nunca antes ouvidas.

  \bigskip

  \quoteRef{Saṃyutta Nikāya 56.11}%
}

\chapter{A Primeira Nobre Verdade}

\section{Existe o Sofrimento}

A Primeira Nobre Verdade é composta por três fases: “Existe o sofrimento,
\emph{dukkha}. \emph{Dukkha} deve ser compreendido. \emph{Dukkha} foi
compreendido.”

É um ensinamento muito prático, expresso numa fórmula, fácil de memorizar. É
também aplicável a toda e qualquer experiência que se possa ter, a tudo o que se
possa fazer ou pensar, relacionado com o passado, o presente ou o futuro.

Sofrimento, \emph{dukkha}, é o elo comum que todos nós partilhamos. Toda a gente
em qualquer lugar sofre. Seres humanos sofreram no passado, na Índia da
antiguidade; sofrem hoje em dia em Inglaterra, e no futuro os seres humanos
também irão sofrer\ldots{} O que é que temos em comum com a Rainha Isabel?
Todos sofremos. O que é que temos em comum com um pobre em Charing Cross?
Sofrimento. Encontra"-se a todos os níveis, desde os seres humanos mais
privilegiados aos mais desesperados e desprivilegiados. É uma ligação que temos
em comum, algo que todos compreendemos.

Quando falamos sobre o sofrimento humano, desperta em nós o sentimento da
compaixão, mas quando damos as nossas opiniões, sobre o que eu penso ou o que
vocês pensam em relação à política e religião, então podemos entrar em guerra.
Há dez anos, em Londres, lembro"-me de ver um filme que mostrava mulheres russas
com bebés e homens russos a levarem os seus filhos a piqueniques, tentando
retratar os russos como seres humanos. Na altura, esta representação do povo
russo era pouco usual, porque a maior parte da propaganda no Ocidente
retratava"-os como monstros ou seres reptilianos de coração gelado, e por esse
motivo nunca pensava neles como seres humanos. Se se quiser eliminar pessoas tem
de se mostrá"-las da pior forma. Não é tão fácil eliminar pessoas se se reconhecer que
estas sofrem da mesma forma que nós. Temos de pensar que elas não têm coração,
que são imorais, más, sem qualquer valor e que o melhor é vermo"-nos livres
delas. Temos de pensar que elas são o mal e que é bom livrarmo"-nos do mal. Com
esta atitude, podemos sentir"-nos justificados ao bombardeá"-las e metralhá"-las
mas, se tivermos em mente o sofrimento como elo comum, isso torna"-nos incapazes
de agir dessa forma.

A Primeira Nobre Verdade não é uma desagradável afirmação metafísica, que apenas
nos diz que tudo é sofrimento. É importante notar que existe uma diferença entre
a doutrina metafísica, em que se faz uma afirmação acerca do Absoluto, e a Nobre
Verdade que é uma reflexão. A Nobre Verdade é uma verdade para ser reflectida,
não é um absoluto, não é O Absoluto. É neste ponto que os Ocidentais se sentem
bastante confusos porque interpretam esta Nobre Verdade como um tipo de verdade
metafísica do Budismo, mas na realidade, nunca houve a intenção de esta ser tal
coisa.

Pode"-se constatar que a Primeira Nobre Verdade não é uma afirmação absoluta,
pois sabe"-se que a Quarta Nobre Verdade é o caminho para o fim do sofrimento.
Não se pode ter sofrimento absoluto e depois ter um caminho para sair dele, ou
será que se pode? Isso não faz sentido. No entanto, algumas pessoas pegam na
Primeira Nobre Verdade e dizem que o Buddha ensinou que tudo é sofrimento.

A palavra Pāli, \emph{dukkha}, significa “incapaz de satisfazer” ou “não ser
capaz de aguentar algo”, ou seja, transiente, incapaz de nos preencher
verdadeiramente ou de nos tornar felizes. O mundo sensorial é assim, uma
vibração na natureza. Seria de facto terrível se encontrássemos satisfação no
mundo dos sentidos, porque então nunca iríamos procurar nada para além dele, ficaríamos
limitados. No entanto, ao despertarmos para \emph{dukkha}, começamos a procurar
a saída para deixarmos de estar constantemente presos à consciência sensorial.

\section{Sofrimento e Identificação Pessoal}

É importante reflectir na construção da frase da Primeira Nobre Verdade, que é
expressa de uma forma bem clara: “O sofrimento existe”, em vez de “Eu sofro”.
Psicologicamente falando, essa reflexão é exposta de uma forma muito mais hábil.
Temos a tendência de interpretar o nosso sofrimento como “Eu estou mesmo a
sofrer. Sofro muito e não quero sofrer”. É assim que pensamos, é desta forma que
a nossa mente está condicionada.

“Eu estou a sofrer” transmite"-nos sempre a sensação de que “Sou alguém que sofre
bastante. Este sofrimento é meu. Eu tenho sofrido bastante na minha vida”. E
assim começa todo o processo de identificação com o nosso eu e a nossa memória,
lembramo"-nos do que aconteceu quando éramos crianças\ldots{} e por aí fora.

Mas reparemos, não estamos a dizer que existe alguém que tem sofrimento. Já não
se trata de sofrimento pessoal quando o vemos como “o sofrimento existe”. Deixa
de ser: “Ah coitado de mim, porque é que eu tenho de sofrer tanto? O que é que
fiz para merecer isto? Porque é que tenho de envelhecer? Porque é que tenho de
ter amargura, dor, lamentação e desespero? Não é justo! Eu não quero isto. Só
quero felicidade e segurança”.

Este tipo de pensar nasce da ignorância, o que complica tudo e dá origem a
problemas de personalidade.

Para podermos abandonar o sofrimento temos de primeiro admiti"-lo na consciência.
Mas na meditação budista esta admissão não parte da posição de “Eu estou a
sofrer” mas sim de “O sofrimento está presente”, pois não estamos a tentar
identificar"-nos com o problema mas, simplesmente a reconhecer que ele existe.
Pensar em termos de “Estou zangado; zango"-me muito facilmente; como ver"-me livre
disto?”, não revela grande sabedoria pois tudo isto desperta em nós uma série de
pressupostos sobre a existência de um \emph{eu}, tornando muito difícil obter
qualquer perspectiva sobre o assunto. Torna"-se muito confuso porque a percepção
dos \emph{meus} problemas ou dos \emph{meus} pensamentos, leva"-nos facilmente a
reprimir ou a fazer juízos de valor acerca do assunto e a criticarmo"-nos a nós
próprios. Em vez de observar, testemunhar e compreender as coisas como elas são,
temos a tendência de nos apegarmos e identificarmos. Quando simplesmente
reconhecemos que existe esta sensação de confusão, que existe este egoísmo ou
raiva, então surge uma reflexão honesta sobre a forma como as coisas são, pois
removemos todas as ideias preconcebidas ou pelo menos não as valorizamos.

Assim sendo, não nos devemos apegar a estas coisas como se fossem falhas
pessoais, mas continuar a contemplá"-las como sendo impermanentes,
insatisfatórias e impessoais. Continuar a reflectir, observando"-as como
realmente são. A tendência é sempre para ver a vida a partir da perspectiva de
que estes são os meus problemas, e de que estou a ser muito honesto e dinâmico
em admitir tal coisa. E na nossa vida reafirmamos isso mesmo, pois continuamos a
funcionar a partir dessa ideia errada. Mas mesmo esse ponto de vista é
impermanente, insatisfatório e “não eu”.

“O sofrimento existe” é um reconhecimento claro e preciso de que neste momento
existe uma certa sensação de descontentamento, que pode ir desde a angústia e
desespero a uma suave irritação; \emph{dukkha} não significa necessariamente
sofrimento severo. Não temos de ser brutalizados pela vida; não temos
necessariamente de ter vindo de Auschwitz ou Belsen para podermos dizer que o
sofrimento existe. Até a Rainha Isabel pode dizer, “O sofrimento existe”.
Estou certo de que ela tem momentos de grande angústia e desespero ou, pelo
menos, de irritação.

O mundo dos sentidos é uma experiência sensível. Significa que se está sempre a
ser exposto ao prazer, à dor e à dualidade do \emph{saṃsāra}. É como estar em
algo que é muito vulnerável, sentindo tudo aquilo que possa entrar em contacto
com estes corpos e os seus sentidos. É assim, esse é o resultado do nascimento.

\section{Negação do Sofrimento}

O sofrimento é algo do qual normalmente não queremos saber, tudo o que queremos é
ver"-nos livres dele. Assim que surge algo inconveniente ou desagradável, a
tendência do ser não iluminado é querer livrar"-se disso ou suprimi-lo. Podemos
observar como a sociedade moderna se encontra tão embrenhada em procurar
prazeres e delícias naquilo que é novo, excitante e \mbox{romântico}. Temos tendência para
colocar ênfase na beleza e prazeres da juventude, enquanto que o lado feio da
vida, a velhice, doença, morte, aborrecimento, desespero e depressão são
colocados de parte.

Quando nos deparamos com algo de que não gostamos, tentamos
ver"-nos livres disso na procura de algo de que gostamos. Se nos sentimos
aborrecidos vamos logo fazer algo interessante, se sentimos medo tentamos
encontrar segurança. Isto é perfeitamente normal. Estamos associados com o
princípio de prazer/dor de atracção/repulsão. Assim, se a mente não está atenta
e receptiva torna"-se selectiva, seleccionando aquilo de que gosta e tentando
suprimir aquilo de que não gosta. Grande parte da nossa vivência tem de ser
suprimida, porque muito daquilo com que estamos inevitavelmente envolvidos é de
certa forma desagradável.

Se surge algo desagradável, dizemos “Foge!”, se alguém se atravessa no nosso
caminho, dizemos “Mata!”. Os nossos governos têm frequentemente a tendência para fazer
isto\ldots{} Se pensarmos no tipo de pessoas que governam os nossos países é
preocupante não é? Elas ainda são bastante ignorantes, não iluminadas. Mas é
assim que funciona, a mente ignorante pensa em exterminação: “Olha um mosquito,
mata"-o!”, “Estas formigas estão a apoderar"-se da cozinha; dá"-lhes com o
insecticida!”. Em Inglaterra temos uma companhia chamada “Rent"-o-kill”. Não sei
se é um tipo de máfia britânica ou não, mas especializa"-se em eliminar pestes –
seja qual for a forma que queiram interpretar a palavra “pestes”.

\section{Moralidade e Compaixão}

É por esse motivo que temos de ter leis como, “Eu abstenho"-me de matar
intencionalmente”, porque o nosso instinto natural é o de matar: se algo aparece no
nosso caminho, mata"-se. Podemos observar isso no reino animal. Somos criaturas
bastante predadoras; pensamos que somos civilizados, mas literalmente temos uma
história bastante sangrenta. É preenchida por inúmeras chacinas e justificações
para todo o tipo de injustiças para com os outros seres humanos, já para não
falar nos animais e, tudo isto, devido a esta ignorância básica, esta mente
humana que sem reflectir nos diz para aniquilar o que está no nosso caminho.

No entanto, ao usar a reflexão, estamos a mudar esta situação; estamos a
transcender esse padrão animal, básico e instintivo. Não somos apenas marionetas
cumpridoras das leis da sociedade, com medo de matar por termos medo de ser
punidos. Agora estamos realmente a começar a ser responsáveis. Respeitamos a
vida das outras criaturas, até mesmo a vida dos insectos e das criaturas de que
não gostamos. Jamais alguém irá gostar de mosquitos ou formigas, mas podemos
reflectir sobre o direito que eles têm de viver. Isto é uma reflexão da mente e
não somente uma reacção: “Onde está o insecticida?”. Eu também não gosto de ver
formigas no meu chão; a minha reacção inicial é, “onde está o insecticida?”. Mas
então, a mente reflexiva, mostra"-me que ainda que estas criaturas me estejam a
irritar e que eu preferisse que elas desaparecessem, elas têm o direito de existir.
Esta é uma reflexão da mente humana.

O mesmo pode ser aplicado a estados mentais desagradáveis. Assim, se de cada vez
que se sentir raiva, em vez de se dizer “Ora lá estou eu zangado outra vez!”,
deve"-se reflectir “Existe raiva”. Tal como com o medo - se o virmos como o medo
que tenho da minha mãe, ou o medo do meu pai, ou o medo do cão ou o meu medo,
tudo se transforma numa teia de diferentes criaturas relacionadas de algumas
maneiras e não relacionadas de outras, tornando"-se difícil haver qualquer tipo
de verdadeiro entendimento. E, no entanto, o medo deste ser e o medo daquele cão
vadio é exactamente o mesmo. “Existe medo”, é apenas isso. O medo que eu já
senti não é em nada diferente do medo dos outros e é assim que temos compaixão
até para com cães vadios velhos. Compreendemos que o medo é tão horrível para os
cães vadios como para nós. Quando um cão leva um pontapé de uma bota pesada e
vocês levam um pontapé de uma bota pesada, a sensação de dor é a mesma. Dor é
somente dor, frio é somente frio, raiva é somente raiva. Não é “minha”, mas simplesmente
“existe dor”. Esta é uma forma inteligente de pensar, que nos ajuda a ver as
coisas de forma mais clara, em vez de reforçar a ideia da personalidade.
Resultando do reconhecimento do estado de sofrimento, que o sofrimento existe,
surge então, a segunda compreensão desta Primeira Nobre Verdade: “Deve ser
compreendido”. Este sofrimento deve ser investigado.

\section{Investigação do Sofrimento}

Encorajo"-vos a tentar compreender \emph{dukkha}, o sofrimento, a observar
honestamente e aceitá"-lo com confiança. Tentem compreendê"-lo quando estiverem a
sentir dor física, desespero e angústia ou ódio e aversão, ou qualquer que seja
a forma que tome, qualquer que seja a sua qualidade, quer ele seja extremo ou
suave. Isto não quer dizer que para serem iluminados têm de ser
miseráveis, deixarem que vos tirem tudo ou serem torturados; significa, ser capaz
de olhar para o sofrimento e compreendê"-lo, nem que seja ainda com uma leve
sensação de descontentamento.

É fácil encontrar um bode expiatório para os nossos problemas. “Se a minha mãe
me tivesse realmente amado ou se todos aqueles à minha volta tivessem sido
verdadeiramente sábios e totalmente dedicados a tentarem proporcionar"-me um
ambiente perfeito, então, eu não teria os problemas emocionais que tenho agora”.
Isto é mesmo tolice! No entanto é desta forma que algumas pessoas vêem o mundo,
pensando que estão confusos e miseráveis porque não receberam o que seria justo.
Mas com esta fórmula da Primeira Nobre Verdade, ainda que tenhamos tido uma vida
muito miserável, aquilo que estamos a observar não é o sofrimento que vem de
fora, mas aquilo que criamos nas nossas mentes à volta desse factor. Isto é um
despertar na pessoa, um despertar para a verdade do sofrimento. E é uma Nobre
Verdade porque já não culpa os outros pelo sofrimento que sentimos. Desta forma,
a abordagem budista é singular em relação a outras religiões, porque se enfatiza
o caminho para deixar o sofrimento através da sabedoria, libertação de toda a
ilusão, em vez da obtenção de algum estado de felicidade ou união com o Supremo.

Não estou a dizer que os outros nunca são a fonte da nossa frustração e
irritação, mas o que estamos a apontar com este ensinamento é a nossa
reacção para com a vida. Se alguém estiver a ser mau para vós ou, propositada e
malevolamente, a tentar fazer"-vos sofrer, e se pensarem que é essa pessoa que
vos está a fazer sofrer, é porque esta Primeira Nobre Verdade ainda não foi compreendida.
Ainda que essa pessoa vos esteja a arrancar as unhas ou a fazer"-vos outras coisas
horríveis, enquanto pensarem que estão a sofrer por causa dela, não
perceberam esta Primeira Nobre Verdade. Perceber o sofrimento é ver claramente
que o verdadeiro sofrimento é a nossa reacção à pessoa que nos está a arrancar as unhas:
“Eu odeio"-te,” - isto é o sofrimento. Ter as unhas arrancadas é doloroso, mas o sofrimento 
envolve “Eu odeio"-te” e “Como é que me podes fazer isto” e “Eu nunca te perdoarei”.

Todavia não esperem que alguém vos arranque as unhas para praticarem a
Primeira Nobre Verdade. Testem"-na com pequenas coisas, como por exemplo, quando
alguém é insensível, mal"-educado ou ignorante para convosco. Se estão a sofrer
porque essa pessoa vos fez alguma desfeita ou vos ofendeu de alguma forma, podem
praticar com isso. Na vida diária existem muitas ocasiões em que podemos
sentir"-nos ofendidos ou zangados. Podemos sentir"-nos irritados simplesmente pela
forma como alguém anda ou pela sua aparência (pelo menos eu posso). Por vezes
apercebemo"-nos da aversão surgindo em nós, simplesmente devido à forma como
alguém anda ou porque não faz algo que deveria fazer. Podemos irritar"-nos com
esse tipo de coisas. A pessoa na realidade não nos fez nada de mal, não nos
arrancou as unhas, mas ainda assim sofremos. Se não conseguirmos enfrentar o
sofrimento nestas situações simples, nunca seremos capazes de ser tão heróicos
perante alguém que nos esteja realmente a arrancar as unhas!

Trabalhamos com as pequenas insatisfações da vida quotidiana. Olhamos para a
forma como podemos ser magoados, ofendidos ou irritados pelos vizinhos, por
pessoas com quem vivemos, pela Sra Tatcher, pela forma como as coisas são ou por
nós próprios. Sabemos que este sofrimento deve ser compreendido. Praticamos
olhando realmente para o sofrimento como um objecto e compreendendo: “Isto é
sofrimento”. Assim temos a reveladora compreensão do sofrimento.

\section{Prazer e Descontentamento}

\enlargethispage{\baselineskip}

Podemos investigar: Até onde nos trouxe esta indulgência pela procura dos
prazeres? Há várias décadas que isto se perpetua, mas será que a humanidade está
mais feliz por isso? Parece que hoje em dia nos foi dada a liberdade de fazermos
tudo aquilo que queremos com drogas, sexo, viagens e por aí fora, tudo é
permitido e nada é proibido. De facto, para se ser marginalizado tem de se
chegar a fazer algo realmente obsceno e verdadeiramente violento. Mas será que o
facto de podermos seguir os nossos impulsos livremente nos tornou mais felizes,
mais descontraídos e mais satisfeitos? Na realidade, isso tem"-nos tornado muito
mais egoístas; não pensamos como as nossas acções podem vir a afectar os outros.
Geralmente pensamos só em nós próprios: Eu e a \emph{minha} felicidade, a
\emph{minha} liberdade e os \emph{meus} direitos. Assim torno"-me num tremendo
chato, uma fonte de imensa frustração, irritação e infelicidade para as pessoas
à minha volta. Se pensar que posso fazer e dizer tudo aquilo que quero, mesmo à
custa dos outros, então torno"-me uma pessoa que nada mais é do que um
aborrecimento para a sociedade.

Quando a sensação “de aquilo que eu quero” e “de aquilo que eu penso que deve ser ou
não deve ser” surge, e nos queremos deliciar com todos os prazeres da vida,
inevitavelmente ficamos contrariados, pois a vida parece tão desesperante e
tudo parece correr mal. A vida põe"-nos em turbilhão, correndo de um lado para o
outro num estado de medo e de desejo. E mesmo quando conseguimos tudo o que
queremos, pensamos que nos falta algo, que algo ainda está incompleto. Assim,
mesmo quando a vida está a correr pelo melhor, ainda existe esta sensação de
sofrimento, de algo ainda a ser feito, um tipo de dúvida ou medo a
assombrar"-nos.

Por exemplo, sempre gostei de paisagens bonitas. Certa vez, durante um retiro
que conduzi na Suíça, levaram"-me a ver umas montanhas muito bonitas. Apercebi"-me
que, perante tanta beleza, havia sempre presente uma sensação de angústia na
minha mente. Perante esta corrente contínua de bonitas paisagens, tive a
sensação de querer abraçar tudo, de a todo o momento ter de me manter bem alerta
para assim absorver tudo aquilo com os meus olhos. Estava mesmo a esgotar"-me!
Ora, isso foi \emph{dukkha}, não foi?

Noto que se faço algo sem prestar atenção - ainda que seja algo tão inocente
como olhar para uma bela montanha -- e se o faço somente a projectar"-me para fora
na tentativa de agarrar algo, traz"-me sempre uma sensação desagradável. Como é
que se pode reter a beleza da Jungfrau e da Eiger? A melhor solução é tirar uma
fotografia, tentar captar tudo num pedaço de papel. Isso é \emph{dukkha}; se se
quer conservar algo bonito porque não se quer separar dele, isso é sofrimento.
Ter de estar presente em situações de que não se gosta também é sofrimento.

Por exemplo, nunca gostei de viajar de metro em Londres. Eu reclamava, “Não
quero ir de metro, as estações são muito sujas e cheias de \emph{posters}
horríveis. Não quero ser empacotado naqueles comboios minúsculos debaixo do
chão”. Achava isto uma experiência completamente desagradável. Mas prestava
atenção a esta voz que reclamava e lastimava - o sofrimento de não querer algo
que é desagradável. Então, tendo reflectido, deixei de tecer elaborações sobre a
situação, para assim poder ficar só com aquilo que é desagradável e feio sem lhe
adicionar mais sofrimento. Percebi que a situação era assim, e está tudo bem.
Não necessitamos de criar mais problemas, quer estejamos numa estação de metro
muito suja ou a apreciar paisagens bonitas. As coisas são como são, podemos
apreciar e reconhecê"-las na sua constante mudança sem nos apegarmos. Apego é
querermos agarrar e jamais largar algo de que gostamos; querermos ver"-nos livres
de algo de que não gostamos; ou querermos ter algo que não temos.

Também podemos sofrer muito por causa de outras pessoas. Lembro"-me que na
Tailândia costumava ter pensamentos bastante negativos sobre um dos monges. Ele
fazia algo e eu pensava “Ele não devia de fazer isso” ou, se ele dizia qualquer
coisa “Ele não devia dizer isso!” Eu carregava este monge na minha mente e
ainda que eu fosse para qualquer outro lugar, eu pensava nele; a imagem dele
surgia e as mesmas reacções vinham à tona: “Lembras"-te quando ele disse isto e
fez aquilo?” e “Ele não devia ter dito isso e ele não devia ter feito aquilo”.

Quando encontrei um professor como o Ajahn Chah, lembro"-me de querer que ele
fosse perfeito. Eu pensava, “Oh! Ele é um professor maravilhoso, maravilhoso!”.
Mas podendo vir a fazer algo que me desagradasse eu pensava, “Eu não quero que
ele faça nada que me desagrade, porque eu gosto de pensar nele como sendo
maravilhoso”. Era como que dizer, “Ajahn Chah, sê sempre maravilhoso para
comigo. Nunca faças nada que ponha qualquer tipo de pensamento negativo na minha
mente”. Por isso, mesmo quando se encontra alguém que realmente se respeita e
ama, existe ainda o sofrimento do apego. Inevitavelmente, eles irão dizer ou 
fazer algo de que não se gosta ou aprova, causando sempre algum tipo de dúvida 
– isto traz sofrimento.

A certa altura, vários monges americanos vieram para Wat Pah Pong, o nosso
mosteiro no Nordeste da Tailândia. Eles eram muito críticos e parecia que só
viam o que estava errado. Eles não achavam que o Ajahn Chah fosse bom professor
e não gostavam do mosteiro. Eu senti uma grande raiva e ódio surgindo em mim,
porque eles estavam a criticar algo que eu adorava. Eu senti"-me indignado, “Bem,
se vocês não gostam, saiam daqui para fora. Ele é o melhor professor do mundo,
se não conseguem ver isso, então desapareçam!”. Esse tipo de apego, estar
enamorado ou ser devoto, é sofrimento, porque se algo ou alguém que se ama ou
gosta é criticado, sentimo"-nos zangados e ofendidos.

\section{Compreensão Interior nas Situações da Vida}

Por vezes a compreensão inteiror surge nas ocasiões mais inesperadas. Isto aconteceu"-me
quando vivia em Wat Pah Pong. O Nordeste da Tailândia não é dos lugares mais
atraentes ou bonitos do mundo, com as suas florestas e vastas planícies; durante
a estação quente torna"-se extremamente quente. Antes de cada Dia de
Observância\footnote{%
  \emph{Dia de Observância}: (em Pāli: \emph{Uposatha}) um dia sagrado ou
  “sabbath”, ocorre em todos dias de Lua Nova e Lua Cheia. Nestes dias os
  budistas reúnem"-se para ouvir o Dhamma e reafirmam a sua prática em termos de
  preceitos e meditação.} nós tínhamos de varrer as folhas caídas nos caminhos
do mosteiro. As áreas a varrer eram bem vastas. Passávamos a tarde toda debaixo
do sol quente suando e a varrer, com vassouras grosseiras, as folhas para um
monte; esta era uma das nossas tarefas. Eu não gostava de o fazer. Pensava, “Eu
não quero fazer isto. Não vim para aqui para varrer as folhas do chão; Vim para
aqui para me tornar iluminado e em vez disso põem"-me a varrer folhas. Para além
disso, está muito calor e eu tenho uma pele clara; posso apanhar cancro da pele
por estar aqui neste clima quente”.

Numa dessas tardes lá estava eu a sentir"-me verdadeiramente infeliz, pensando “O
que é que estou aqui a fazer? Porque é que vim para aqui? Porque é que estou
aqui?”. E ali fiquei parado com a minha vassoura longa e grosseira, sem energia, 
a sentir pena de mim mesmo e a odiar tudo. Então Ajahn Chah
aproximou"-se, sorriu"-me e disse «Em Wat Pah Pong há bastante sofrimento, não
há?» e continuou a andar. Então pensei, “Porque é que ele disse aquilo? E sabes,
na verdade, não é assim tão mau”. Ele levou"-me a reflectir “Será que varrer as
folhas é mesmo tão desagradável?\ldots{} Não, não é. É algo neutro;
Varrer as folhas, não é bom nem mau\ldots{} E suar é algo assim tão terrível? É
mesmo uma experiência miserável e humilhante? É mesmo assim tão mau como eu
estou a querer fazer parecer?\ldots{} Não, suar não faz mal, é algo
perfeitamente natural. E eu não tenho cancro da pele e as pessoas em Wat Pah
Pong são muito simpáticas. O professor é um homem muito bondoso e sensato. Os
monges têm"-me tratado bem. As pessoas leigas vêm e dão"-me comida e\ldots{}
afinal porque é que eu estou a reclamar?”. Reflectindo acerca da verdade da
minha experiência, pensei “Eu estou bem. As pessoas respeitam"-me, sou bem
tratado. Estou a ser ensinado por pessoas agradáveis num país também agradável.
Não há nada de errado nisto, mas sim em mim; Estou a criar um problema porque não
quero varrer folhas e suar”. E com isto tive uma compreensão. De repente, senti
que havia algo em mim sempre a reclamar e a criticar, que impedia que me
entregasse totalmente a diversas situações.

Outra experiência com a qual aprendi foi o costume de lavar os pés dos monges
mais velhos quando eles regressavam da ronda da mendicância. Depois de
caminharem pelas vilas e arrozais, os seus pés estavam enlameados. Quando o
Ajahn Chah regressava, todos os monges, talvez cerca de vinte ou trinta,
apressavam"-se para o receber e lhe lavar os pés no lava"-pés que havia à entrada
da sala de refeições. Quando vi isto pela primeira vez, pensei “Eu nunca vou
fazer tal coisa!”. E no dia seguinte, assim que o Ajahn Chah apareceu, trinta
monges apressaram"-se para lhe lavar os pés. Eu pensei “Que coisa tão estúpida,
trinta monges a lavarem os pés de um homem. Eu não o faço”. No dia seguinte, a
minha reacção tornou"-se ainda mais violenta\ldots{} trinta monges apressaram"-se e
lavaram os pés do Ajahn Chah\ldots{} “Isto irrita"-me mesmo, estou farto disto! Acho
que é a coisa mais estúpida que alguma vez vi, trinta homens a lavar os pés de
um homem! Provavelmente ele pensa que o merece; é só para lhe reforçar o ego.
Ele deve ter um ego enorme, com todas estas pessoas a lavarem"-lhe os pés todos
os dias. Eu nunca farei tal coisa!”.

Eu estava a começar a ter uma reacção extrema. E ali ficava, sentado,
sentindo"-me miseravelmente zangado. Olhava para os monges e pensava, “Que gente
tão estúpida. Não sei o que estou aqui a fazer”.

Mas então comecei a reflectir; “É mesmo desagradável estar neste estado de
espírito. Será que isto é mesmo algo para me deixar assim tão zangado? Ninguém
me obrigou a fazer tal coisa, está tudo bem; não há nada de errado com trinta
homens a lavarem os pés a outro. Não é imoral, nem mau comportamento e talvez
eles não se importem; talvez eles o queiram fazer, talvez não haja problema
nenhum\ldots{} Talvez eu devesse fazê"-lo!”. E assim na manhã seguinte, trinta e
um monges se apressaram a lavar os pés do Ajahn Chah. Depois disto deixou de
haver qualquer problema. Senti"-me mesmo bem: aquela coisa má em mim tinha
cessado.

Podemos reflectir sobre estas coisas que nos causam indignação e raiva; existe
algo de verdadeiramente errado nelas ou são apenas coisas sobre as quais criamos
\emph{dukkha}? Desta forma, começamos a perceber os problemas que criamos nas
nossas vidas e nas vidas das pessoas à nossa volta.

Com consciência, estamos dispostos a suportar tudo o que a vida nos dá; a
excitação e o aborrecimento, a esperança e o desespero, o prazer e a dor, o
fascínio e a fadiga, o princípio e o fim, o nascimento e a morte. Dispomo"-nos a
aceitar tudo na mente em vez de apenas absorver o que nos é agradável e suprimir
o que é desagradável. O processo que conduz à sabedoria passa por \emph{dukkha},
observando, aceitando e reconhecendo \emph{dukkha} em todas as suas formas.
Então deixa"-se naturalmente de reagir da forma habitual, de ser indulgente
na satisfação de todos os desejos ou de os suprimir. E por essa razão,
consegue"-se suportar melhor o sofrimento, tornando"-nos mais pacientes na sua presença.

Estes ensinamentos são exteriores à nossa experiência pessoal. Eles são, de
facto, reflexões da nossa verdadeira experiência e não complicadas questões
intelectuais. Assim, há que colocar energia no seu desenvolvimento e não ficar
encalhado na rotina habitual. Quantas vezes, é que se têm de se sentir culpados
por causa do aborto que fizeram, ou dos erros que cometeram no passado? Será que
têm de passar todo o vosso tempo a regurgitarem as coisas que aconteceram na vida e
a entregarem"-se a infinitas especulações e análises? Algumas pessoas tornam"-se
personalidades complicadas. Se apenas se entregarem às memórias, pontos de vista
e opiniões, ficarão para sempre prisioneiras do mundo, e jamais, de forma
alguma, o transcenderão.

Podem abandonar este pesado fardo se estiverem dispostos a utilizar os
ensinamentos com perícia. Digam a vós próprios: “Não me vou envolver mais nisto;
Recuso"-me a participar neste jogo. Não me vou deixar levar mais por este estado
de espírito”. Comecem a colocar"-se na posição de quem sabe: “Sei que isto é
\emph{dukkha}; \emph{dukkha} existe”. É muito importante que tomem a resolução
de ir ao encontro do sofrimento e que depois o tolerem. Somente examinando e
confrontando o sofrimento deste modo é que podemos esperar ter um grande momento
de clareza: “Este sofrimento foi compreendido”.

Estes são os três aspectos da Primeira Nobre Verdade. Esta é a fórmula que temos
de usar e aplicar na reflexão sobre as nossas vidas. Sempre que sentirem
sofrimento, reconheçam"-no primeiro “O sofrimento existe”, depois “Ele deve ser
compreendido” e finalmente “Ele foi compreendido”. Este entendimento do
\emph{dukkha} é a realização clara da Primeira Nobre Verdade.

