\chapterNote{%
  O que é a Nobre Verdade do Caminho que Conduz à Cessação do Sofrimento?

  É este Nobre Caminho Óctuplo, que é como dizer, Entendimento Correcto,
  Intenção Correcta, Discurso Correcto, Acção Correcta, Meio de Subsistência Correcto,
  Esforço Correcto, Consciência Correcta e Concentração Correcta.

  Existe esta Nobre Verdade do Caminho que Conduz à Cessação do Sofrimento: tal
  foi a visão, revelação, sabedoria, verdadeiro conhecimento e luz que em mim surgiram,
  acerca de coisas nunca antes ouvidas. 
  
  Esta Nobre Verdade deve ser penetrada cultivando o Caminho\ldots{}

  Esta Nobre Verdade foi penetrada, cultivando o Caminho: tal foi a visão,
  revelação, sabedoria, verdadeiro conhecimento e luz que em mim surgiram,
  acerca de coisas nunca antes ouvidas.

  \bigskip

  \quoteRef{Saṃyutta Nikāya 56.11}%
}

\chapter{A Quarta Nobre Verdade}

\pagestyle{topbottomcorner}

\section{Existe o Nobre Caminho}

A Quarta Nobre Verdade, tal como as primeiras três, é composta por três aspectos.
O primeiro é: “Existe o Caminho Óctuplo, ou \emph{atthangika magga} – o caminho
para sair do sofrimento”. É também chamado de \emph{ariya magga}, o
\emph{Ariyan} ou Nobre Caminho. O segundo aspecto é que: “Este caminho deve ser
desenvolvido”. E o último aspecto é que: “Este caminho foi plenamente desenvolvido”.

O Caminho Óctuplo é apresentado numa sequência, começando com Entendimento
Correcto (ou perfeito), \emph{sammā diṭṭhi}, seguindo"-se a Intenção ou Aspiração
Correcta, \emph{sammā saṅkappa}; estes dois elementos do caminho são agrupados
como Sabedoria (\emph{paññā}). Compromisso moral (\emph{sīla}) fluí da
sabedoria, que por sua vez abrange Discurso Correcto, Acção Correcta e Meio de
Subsistência Correcto – por vezes referidos como, \emph{sammā vācā}, \emph{sammā
  kammanta} e \emph{sammā ājīva}.

Depois temos Esforço Correcto, Consciência Correcta e Concentração Correcta,
\emph{sammā vāyāma}, \emph{sammā sati} e \emph{sammā samādhi}, que flúem
naturalmente do compromisso moral. Estes três últimos fornecem equilíbrio
emocional. Falam do coração – o coração que está livre da identificação com o Eu
e livre do egoísmo. Com o Esforço Correcto, Consciência Correcta e Concentração
Correcta, o coração é puro, livre de impurezas e contaminações. Quando o coração
está puro, a mente está em paz. A Sabedoria (\emph{paññā}), ou Entendimento
Correcto e Aspiração Correcta, nascem do coração puro. Isto leva"-nos de volta ao
início.

\clearpage

Assim, estes são os elementos do Caminho Óctuplo, agrupados em três secções:

\bigskip

\noindent
Sabedoria (\emph{paññā})

\bigskip

\begin{packedenumerate}

\item Entendimento Correcto (\emph{sammā diṭṭhi})
\item Aspiração Correcta (\emph{sammā saṅkappa})

\end{packedenumerate}

\bigskip

\noindent
Moralidade (\emph{sīla})

\bigskip

\begin{packedenumerate}
\setcounter{enumi}{2}

\item Discurso Correcto (\emph{sammā vācā})
\item Acção Correcta (\emph{sammā kammanta})
\item Meio de Subsistência Correcto (\emph{sammā ājīva})

\end{packedenumerate}

\bigskip

\noindent
Concentração (\emph{samādhi})

\bigskip

\begin{packedenumerate}
\setcounter{enumi}{5}

\item Esforço Correcto (\emph{sammā vāyāma})
\item Consciência Correcta (\emph{sammā sati})
\item Concentração Correcta (\emph{sammā samādhi})

\end{packedenumerate}

\bigskip

O facto de aqui os apresentarmos de forma ordenada não significa que os mesmos
acontecem desta forma linear ou nesta sequência – podem acontecer em simultâneo.
Quando falamos sobre do Caminho Óctuplo, dizemos “Primeiro temos Compreensão
Correcta, depois temos Aspiração Correcta, depois\ldots{}” Mas na realidade, seja
qual for a forma, ensina"-nos naturalmente a reflectir acerca da importância de
aceitarmos a responsabilidade, sobre aquilo que dizemos e fazemos nas nossas
vidas.

\clearpage

\section{Entendimento Correcto}

O primeiro aspecto do Caminho Óctuplo é o Entendimento Correcto que nasce da
realização das três primeiras Nobres Verdades. Se tiverem esse vislumbre, terão
perfeito entendimento acerca do Dhamma, o entendimento de que, “Tudo o que está
sujeito a surgir está sujeito a cessar”. É tão simples como isso. Não precisam
de passar muito tempo a ler para compreender as palavras, “Tudo o que está
sujeito a surgir está sujeito a cessar”. Mas leva bastante tempo para a maioria
das pessoas perceber verdadeiramente o que as palavras significam no seu sentido
mais profundo, em vez de terem um simples entendimento racional.

Pondo isto em termos mais coloquiais, a verdadeira realização é o conhecimento
que surge do âmago e não exclusivamente das ideias. Deixa de ser “Eu
\emph{penso} que sei”, ou “Ah, sim! Isso parece ser uma ideia sensata. Concordo.
Gosto dessa ideia”. Este tipo de compreensão continua a ser mental,
superficial, ao passo que a verdadeira compreensão é bem mais profunda. É a
verdadeira sabedoria na qual as dúvidas deixam de ser um problema.

Esta compreensão profunda provém da realização das nove fases previamente
mencionadas. Assim sendo, temos uma sequência que conduz ao Entendimento
Correcto das coisas tal como elas são, nomeadamente que “Tudo o que está sujeito
a surgir está sujeito a cessar”, e não é o eu. Com Correcto Entendimento
abandona"-se a ilusão de um eu ligado a condições mortais. O corpo continua a
existir, continuam a existir pensamentos e sentimentos, mas estes são
simplesmente aquilo que são, deixa de existir a crença de que se é o corpo, os
pensamentos ou os sentimentos. A ênfase está na frase, “As coisas são como são”.
Não estamos a sugerir que as coisas não são absolutamente nada, ou que não são
aquilo que são. São exactamente aquilo que são e nada mais. Mas quando somos
ignorantes, quando ainda não compreendemos estas verdades, pensamos que as
coisas são mais do que aquilo que são. Acreditamos em todo o tipo de coisas e
criamos todo o tipo de problemas em redor das condições que sentimos.

Grande parte da angústia e do desespero humano provém do exagero que nasce da
ignorância do momento. É triste constatar que a miséria, a angústia e o
desespero da humanidade estão baseados numa ilusão; o desespero é vazio e sem
sentido. Quando se percebe isto sente"-se uma compaixão infinita por todos os
seres. Como é possível odiar alguém ou guardar rancor ou condenar quem quer que
seja que se encontre prisioneiro nesta teia de ignorância? Toda a gente é
influenciada a fazer aquilo que faz devido à visão incorrecta das coisas.

\sectionBreak

Com a prática de meditação, sentimos alguma tranquilidade, uma certa calma em
que a mente “abranda o passo”. Quando olhamos para algo, com uma mente calma,
por exemplo uma flor, estamos a olhar para ela tal como é. Quando não existe
qualquer apego, nada a ganhar ou nada de que nos queiramos libertar, então se
aquilo que vimos, ouvimos ou sentimos através dos sentidos for bonito, é
verdadeiramente bonito. Não estamos a comparar, a criticar ou a tentar possuir;
encontramos encanto e alegria na beleza à nossa volta e não necessitamos de lhe
adicionar mais nada. É exactamente aquilo que é.

A beleza relembra"-nos a pureza e a verdade. Não a devemos ver como um engodo
para nos iludir: “Estas flores estão aqui só para me atraírem e me iludirem”,
essa é a atitude do velho meditador rabugento! Quando olhamos com um coração
puro para um indivíduo do sexo oposto, apreciamos a sua beleza sem o desejo de
qualquer tipo de contacto ou posse. Quando não existe qualquer desejo ou
interesse pessoal podemos deliciar"-nos na beleza das outras pessoas, tanto
homens como mulheres. Existe honestidade: as coisas são como são. Isto é o que
queremos dizer com a palavra libertação ou \emph{vimutti}, em Pāli. Estamos
libertos destes laços que distorcem e corrompem a beleza à nossa volta, como por
exemplo os nossos corpos. No entanto, as nossas mentes podem tornar"-se tão
corruptas, negativas, depressivas e obcecadas com certas coisas, que deixamos de
as ver tal como são. Se não tivermos Entendimento Correcto vemos tudo através de
véus e filtros cada vez mais densos.

O Entendimento Correcto deve ser desenvolvido através da reflexão, utilizando os
ensinamentos do Buddha. O \emph{Sutta Dhammacakkappavattana} é um ensinamento
bastante interessante para ser contemplado e usado como referência para
reflexão. Mas podemos usar outros \emph{suttas da Tipiṭaka},\footnote{%
  \emph{Tipiṭaka}: literalmente “três cestos”, a colecção das escrituras
  Budistas, classificadas de acordo com Sutta (Discursos), Vināya (Disciplina ou
  Treino) e Abhidhamma (Metafísica).} nomeadamente aqueles relacionados com
\emph{paṭicca"-samuppāda}.\footnote{%
  \emph{Paṭicca"-samuppāda}: génese dependente, a apresentação por etapas de como
  o sofrimento surge dependendo do grau de ignorância e de desejo e, de como
  termina com a sua cessação.}

Este é um ensinamento fascinante para reflexão.
Se conseguirem contemplar tais ensinamentos, podem observar claramente a
diferença entre a forma como as coisas são, como Dhamma, e o ponto em que
deixamos de criar ilusões acerca da forma como as coisas são. É por esse motivo
que temos de estabelecer total e consciente atenção nas coisas tal como elas
são. Se existir o conhecimento das Quatro Nobres Verdades existe também o
Dhamma.

Com o Entendimento Correcto tudo é visto como Dhamma, como por exemplo: estamos
aqui sentados\ldots{} Isto é Dhamma. Não pensamos neste corpo ou na mente como
sendo a nossa personalidade com todos os seus pontos de vista e opiniões, todos
os pensamentos condicionados e reacções, que adquirimos devido à nossa
ignorância. Reflectimos neste preciso momento, aqui e agora como: “Isto é tal
como é. Isto é Dhamma”. Trazemos à mente o entendimento de que esta forma física
é simplesmente Dhamma. Não é o eu; não é pessoal.

Observamos também a sensibilidade desta forma física como Dhamma em vez de algo
pessoal: “Sou sensível”, ou “Não sou sensível”; “Tu não és sensível para comigo.
Quem é o mais sensível?”\ldots{} “Porque é que sentimos dor? Porque é que Deus
criou a dor; porque é que ele não criou só prazer? Porque é que existe tanto
sofrimento e infelicidade no mundo? É injusto. Pessoas morrem e temos de nos
separar das pessoas que amamos; a angústia é terrível”.

Que Dhamma é que existe nisso? É tudo identificação pessoal: “Coitado de mim. Eu
não gosto disto, não quero que seja desta forma. Quero segurança, felicidade,
prazer e o melhor de tudo; não é justo que “Eu” não tenha estas coisas. Não é
justo que os meus pais não fossem \emph{Arahants} quando eu vim ao mundo. Não é
justo que eles nunca escolham \emph{Arahants} para Primeiros"-Ministros de
Inglaterra. Se tudo fosse justo eles elegeriam \emph{Arahants} para
Primeiros"-Ministros!”

Estou a tentar levar esta ideia de que “Isto não é justo, isto não está certo”
ao ponto do absurdo, na tentativa de mostrar como nós esperamos que Deus nos dê
tudo aquilo que necessitamos para sermos felizes. Isso é o que as pessoas
geralmente pensam ainda que não o digam. Mas quando reflectimos, vemos que “Tudo
é da forma que deve ser. A dor é assim. O prazer é desta forma. A consciência cognitiva 
é assim”. Sentimos. Respiramos. Temos aspirações. Quando reflectimos vemos a nossa
própria humanidade tal como é. Deixamos de nos relacionar com ela de forma
pessoal ou de culpar quem quer que seja, porque as coisas não são exactamente
como nós gostamos ou queremos. São como devem ser e nós somos como somos. Podem
questionar"-se, porque é que não somos todos iguais, com a mesma cólera, o mesmo
egoísmo e a mesma ignorância, sem todas as variações e permutações. Ainda que
consigamos resumir a experiência humana aos seus elementos básicos, cada um de
nós tem o seu \emph{kamma}\footnote{%
  \emph{Kamma}: (em Sânscrito: \emph{karma}) Acção de causa e efeito. Causa que
  é criada e recriada pelos impulsos habituais, vontade própria ou energias
  naturais.} para viver, as suas próprias obsessões e tendências, que são sempre
diferentes em qualidade e quantidade em relação aos outros.

Porque é que não podemos ter igualdade, termos exactamente o mesmo de tudo, do
bom e do mau, sermos todos exactamente iguais em termos de aparência, um único
ser andrógeno? Num mundo assim nada seria injusto, diferenças não seriam
permitidas, tudo seria absolutamente perfeito e não existiria a desigualdade.
Contudo, ao reconhecermos o Dhamma, vemos que dentro do reino condicional não
existem duas coisas iguais. Tudo é bastante diferente, com infinitas variáveis e
em constante mudança, e quanto mais tentarmos manipular estas condições de
acordo com as nossas ideias, mais frustrados ficamos. Tentamos criar os outros e
a sociedade, de forma a que se enquadrem na nossa ideia de como as coisas devem
de ser, mas acabamos sempre por ficar frustrados. Com reflexão, compreendemos:
“É assim que deve ser”, as coisas têm de ser desta forma e só podem ser desta
forma.

Ora, isto não é uma reflexão negativa ou fatalista. Não se trata de uma atitude
de “É assim que deve ser e não há mais nada a fazer”' Mas sim uma resposta
bastante positiva, no sentido em que aceitamos o fluir da vida tal como nos é
apresentado. Ainda que não seja aquilo que desejamos, podemos aceitá"-lo e
aprender com ele.

\sectionBreak

Somos seres conscientes e inteligentes com capacidade de memória. Temos uma
linguagem. Durante os últimos milhares de anos, desenvolvemos a capacidade de
raciocínio, inteligência lógica e discriminativa. Aquilo que precisamos de fazer
é tentar perceber como usar estas capacidades, como ferramentas para a
realização do Dhamma, em vez de as vermos como aquisições ou problemas pessoais.
As pessoas que desenvolvem a inteligência discriminativa geralmente usam"-na
contra elas próprias, tornando"-se extremamente críticas de si próprias ao ponto
de se começarem a odiar. Isto acontece porque as nossas faculdades
discriminativas tendem a focar"-se em tudo o que está errado. Discriminação é
isto mesmo: observar como \emph{isto} é diferente \emph{daquilo}. O que acontece
quando usamos este sistema connosco próprios? Uma interminável lista de defeitos
e culpas que nos deixam completamente desesperados.

Quando estamos a desenvolver o Entendimento Correcto usamos a nossa inteligência
para reflexão e contemplação das coisas. Usamos também o nosso estado consciente desperto,
abertos para a forma como as coisas são. Quando reflectimos desta forma
estamos a ter sabedoria e consciência, em simultâneo. Neste caso, estamos a
usar a nossa capacidade discriminativa com sabedoria (\emph{vijjā}) em vez de
ignorância (\emph{avijjā}). Este ensinamento das Quatro Nobres Verdades ajuda"-nos 
a usar a inteligência, a habilidade de contemplar, reflectir e
pensar, de forma sábia e não de forma auto"-destrutiva, egoísta ou rancorosa.

\section{Aspiração Correcta}

O segundo elemento do Caminho Óctuplo é \emph{sammā saṅkappa}, por vezes traduzido como
“Pensamento Correcto”, o pensar de forma correcta. Mas na realidade, possui uma
qualidade mais dinâmica, como por exemplo “intenção”, “atitude” ou “aspiração”.
Eu gosto de usar o termo “aspiração” que de certa forma é bastante significativo
neste Caminho Óctuplo, porque nós aspiramos.

É importante perceber que aspiração não é desejo. Em Pāli a palavra
“\emph{taṇhā}” significa desejo que provém da ignorância, ao passo que
“\emph{saṅkappa}” significa aspiração que não surge da ignorância. Pode"-se
pensar que aspiração é um tipo de \emph{taṇhā}, o desejo de querer ser iluminado
(\emph{bhava taṇhā}), mas \emph{sammā saṅkappa} tem origem no Entendimento
Correcto, o ver claramente. Não é querer tornarmo"-nos algo, nem sequer se
trata do desejo de sermos iluminados. Com o Entendimento Correcto toda essa ilusão
e forma de pensar deixa de fazer qualquer sentido.

A aspiração é um sentimento, intenção, atitude ou movimento dentro de nós. O
nosso espírito eleva"-se, não se afunda - não há desespero! Quando temos
Entendimento Correcto aspiramos à verdade, à beleza e à bondade. \emph{Sammā
  diṭṭhi} e \emph{sammā saṅkappa}, Entendimento Correcto e Aspiração Correcta,
são chamadas de \emph{paññā} ou Sabedoria e formam a primeira das três secções
no Caminho Óctuplo.

\sectionBreak

Podemos reflectir: porque é que nos sentimos insatisfeitos mesmo quando
possuímos o melhor de tudo? Mesmo que tenhamos uma bonita casa, um carro, o
casamento perfeito, filhos bons e inteligentes e tudo o resto, não estamos
completamente felizes e certamente também não estamos satisfeitos quando não
temos todas estas coisas!\ldots{} Se não as temos, podemos pensar, “Bem, se eu
tivesse o melhor, estaria satisfeito”. Mas não estaríamos!

A terra não é o lugar
para nossa satisfação, nem é suposto ser. Quando compreendemos isso, deixamos de
esperar contentamento do planeta Terra, deixamos de fazer tal exigência.

Até percebermos que este planeta não pode satisfazer todos os nossos quereres,
continuaremos a perguntar, “Porque é que não me satisfazes, Terra Mãe?” Somos
como meninos da mamã, constantemente a tentar sugar mais, e a querer que ela nos
nutra e nos torne felizes.

Se estivéssemos contentes não nos questionaríamos sobre as coisas à nossa volta.
No entanto, reconhecemos que existe algo mais para além da terra, debaixo dos
nossos pés; existe algo acima de nós que não conseguimos entender bem. Temos a
capacidade de questionar e ponderar a vida, de compreender o seu significado. Se
quisermos saber qual o significado da vida, não podemos estar satisfeitos só com
bens materiais, conforto e segurança.

E assim sendo, aspiramos a saber a verdade. Podemos pensar que isso é uma forma
de presunção, “Quem é que eu penso que sou? Tão pequeno e insignificante a
tentar descobrir a verdade de tudo”. Mas essa aspiração existe. Se tal não fosse
possível porque é que a teríamos? Considere"-se o conceito da realidade suprema.
Uma verdade absoluta ou suprema é um conceito muito refinado; a ideia de Deus ou
da imortalidade é de facto um pensamento muito refinado. Aspiramos ao
conhecimento dessa derradeira realidade. O nosso lado animal não aspira, não
sabe nada acerca de tais aspirações. Existe em cada um de nós uma inteligência
intuitiva que quer saber: está sempre connosco, mas evitamos compreendê"-la e
valorizá"-la. Geralmente ignoramos ou desconfiamos dela. Especialmente os
materialistas modernos - pensam que se trata apenas de uma fantasia.

Quanto a mim próprio, fiquei verdadeiramente feliz quando compreendi que o
planeta não é a minha verdadeira casa. Foi algo de que sempre suspeitei.
Lembro"-me de pensar, ainda criança, “Acho que não pertenço a este lugar”. Nunca
senti que o planeta Terra fosse o lugar ao qual realmente pertencia, mesmo
antes de ser monge nunca senti que me integrava na sociedade. Para algumas
pessoas isso poderia ser apenas um problema neurótico, mas talvez pudesse ser
aquele tipo de intuição que geralmente as crianças têm. Quando se é inocente, a
mente é bastante intuitiva.

A mente de uma criança está mais intuitivamente em
contacto com forças misteriosas do que a mente da maior parte dos adultos. À
medida que crescemos, somos condicionados a pensar de formas pré"-determinadas e
a ter ideias bem definidas daquilo que é real e daquilo que não é. À medida que
desenvolvemos os nossos egos, a sociedade dita aquilo que é verdadeiro e o que
não é, o que está certo e o que está errado, e assim começamos a interpretar o
mundo através dessas percepções fixas. Uma das coisas que achamos encantador nas
crianças é o facto de elas ainda não o fazerem; elas ainda vêm o mundo com a
mente intuitiva que ainda não está condicionada.

A meditação é uma forma de descondicionar a mente, que nos ajuda a abrir mão de
todas as opiniões extremas e ideias fixas que possuímos. Normalmente aquilo que
é real é posto de parte ao passo que aquilo que não é real prende a nossa
atenção. Ignorância, ou \emph{avijjā}, é mesmo isso.

A contemplação da nossa aspiração humana liga"-nos a algo mais elevado do que
somente ao reino animal ou ao planeta Terra. Para mim essa ligação parece ser
mais verdadeira do que a ideia de que isto é tudo o que existe; de que quando
morremos os nossos corpos apodrecem e nada mais existe. Quando ponderamos e
questionamos acerca do universo em que vivemos, percebemos que é muito vasto,
misterioso e incompreensível. No entanto, quando confiamos abertamente na nossa
mente intuitiva e abdicamos das nossas reacções fixas e condicionadas, podemos
tornar"-nos mais receptivos a coisas que talvez já tenhamos esquecido ou para as
quais nunca nos abrimos antes.

Podemos ter a ideia fixa de que somos uma personalidade, de sermos um homem ou
uma mulher, sermos portugueses, ingleses ou americanos. Tudo isto pode ser bem
verdadeiro para nós, e podemos nos transtornar e zangar por causa delas. Até
estamos dispostos a matar"-nos uns aos outros por causa destas opiniões
condicionadas em que acreditamos e às quais nos apegamos sem nunca sequer as
questionarmos. Sem Aspiração Correcta e Entendimento Correcto, sem \emph{paññā},
nunca conseguiremos ver a verdadeira natureza destas opiniões.

\section{Discurso Correcto, Acção Correcta e Meio de Subsistência Correcto}

\emph{Sīla}, o aspecto moral do Caminho Óctuplo, consiste em Discurso Correcto,
Acção Correcta e Meio de Subsistência Correcto.  Isto significa assumir responsabilidade
pela forma como falamos e termos cuidado com aquilo que fazemos com os nossos
corpos. Quando estou consciente e tenho cuidado, falo de forma apropriada ao lugar e ao
momento; da mesma maneira, actuo ou trabalho de acordo com o momento e o lugar.

Começamos então a compreender que temos de ter cuidado com aquilo que dizemos e
fazemos senão magoamo"-nos constantemente. Acabamos sempre por receber o
resultado de tudo aquilo que dizemos ou fazemos de forma cruel ou injusta.

No passado até podem ter evitado a responsabilidade da mentira, com distracções,
para assim não terem de pensar muito no assunto. Por uns tempos podem esquecer
tudo isso, até que eventualmente a mentira vos apanhe, mas se praticarem sīla,
tudo é mais imediato. Até quando exagero, algo em mim diz, “Não deves exagerar,
deves ter mais cuidado”. Tinha o hábito de exagerar as coisas, é parte da
nossa cultura, algo perfeitamente normal. Mas quando se está consciente, o
efeito da mais pequena mentira é sentido de imediato, porque se está mais
aberto, vulnerável e sensível. Assim, tenham mais cuidado com aquilo que fazem e 
percebam o quão importante é serem responsáveis pelos vossos actos.

O impulso para ajudar os outros é uma forma hábil de Dhamma.\footnote{%
  \emph{Dhamma}: (em Sânscrito: \emph{Dharma}) a lei da verdade universal, a
  natureza ou constituição das coisas.} Se vêem alguém a desmaiar e cair no chão,
uma forma espontânea de Dhamma surge na mente: “Ajuda esta pessoa”, e em seguida
dispõem"-se a ajudá"-la a recuperar os sentidos. Se o fizerem com uma mente vazia,
sem qualquer interesse pessoal, somente por compaixão e por ser aquilo que é
correcto fazer, então, toda essa situação é simplesmente Dhamma, correcto
Dhamma. Não é \emph{kamma} pessoal; não é vosso. Mas se o fizerem por desejo de
ganhar mérito e de impressionar os outros ou porque a pessoa é rica e esperam
receber uma recompensa pela boa acção, então, ainda que a acção seja honrosa,
estão a fazer uma ligação pessoal com a situação e isto reforça a ideia do eu.
Quando fazemos bons trabalhos, motivados pela consciência e pela sabedoria em
vez da ignorância, temos \emph{dhammas} positivos sem \emph{kamma} pessoal.

A ordem monástica foi estabelecida pelo Buddha para que homens e mulheres
pudessem viver uma vida impecável e completamente irrepreensível. Um monge vive
de acordo com um sistema completo de preceitos, chamado de disciplina
\emph{Pattimokkha}. Quando se vive sob esta disciplina, ainda que as acções ou
linguagem sejam descuidadas, pelo menos não deixam fortes impressões. Não se
pode possuir dinheiro e por isso não se pode ir a lugar algum até que se seja
convidado. É-se celibatário. Como se vive da recolha de oferendas, não se mata
quaisquer animais. Nem sequer se colhe flores ou folhas ou se faz qualquer tipo
de acção que possa de alguma forma perturbar o fluir natural; é-se completamente
inofensivo. De facto, na Tailândia tínhamos de trazer sempre connosco filtros de
água para assim pudermos filtrar quaisquer seres vivos que estivessem na água,
tais como as larvas de mosquito.

É totalmente proibido matar intencionalmente seja o que for. Há já vinte e cinco
anos que vivo sob esta regra, não tendo por isso, nenhuma pesada acção kámmica.
Sob esta disciplina, vivemos de uma forma bastante inofensiva e responsável.
Talvez a parte mais difícil seja em relação ao uso da linguagem; os hábitos de
linguagem são os mais difíceis de mudar e de abandonar, mas podemos sempre
melhorá"-los. Com reflexão e contemplação, começamos a ver como é desagradável
dizer idiotices ou simplesmente falar por falar.

Para as pessoas leigas, o Meio de Subsistência Correcto é algo que é desenvolvido à
medida que se começa a perceber quais são as intenções do que se faz. Pode"-se
tentar evitar fazer mal propositadamente a outras criaturas ou ganhar a vida de
forma prejudicial. Pode"-se também evitar ter um meio de subsistência que faça com 
que outras pessoas se tornem dependentes de drogas ou álcool ou algo que possa pôr
em risco o equilíbrio ecológico do planeta.

Assim estes três aspectos, Acção Correcta, Discurso Correcto e Meio de Subsistência
Correcto - surgem na sequência do Entendimento Correcto ou perfeita sabedoria.
Começamos a sentir que queremos viver de uma forma que seja uma bênção para este
planeta, ou pelo menos, que não o maltrate.

Entendimento Correcto e Aspiração Correcta têm definitivamente influência
naquilo que fazemos e dizemos. Assim \emph{paññā}, ou sabedoria, conduz a
\emph{sīla}: Discurso Correcto, Acção Correcta e Meio de Subsistência Correcto.
\emph{Sīla} faz referência às nossas acções e linguagem; com \emph{sīla}
contemos o impulso sexual ou o uso do corpo de forma violenta, não o utilizamos
para matar ou para roubar. Desta forma, \emph{paññā} e \emph{sīla} trabalham
juntas em perfeita harmonia.

\section{Esforço Correcto, Consciência Correcta e Concentração Correcta}

Esforço Correcto, Consciência Correcta e Concentração Correcta referem"-se ao teu
espírito, ao coração. Quando pensamos no espírito, apontamos para o meio do
peito, para o coração. Assim temos \emph{paññā} (a cabeça), \emph{sīla} (o
corpo) e \emph{samādhi} (o coração). Pode"-se usar o próprio corpo como uma
espécie de mapa, o símbolo do Caminho Óctuplo. Os três estão integrados,
trabalhando juntos para a realização e apoiando"-se mutuamente como um tripé.
Nenhum domina o outro nem explora ou rejeita o que quer que seja.

Trabalham juntos: a sabedoria do Entendimento Correcto e da Aspiração Correcta;
depois vem a moralidade, que é o Discurso Correcto, Acção Correcta e Meio de
Subsistência Correcto; depois vem o Esforço Correcto, Consciência Correcta e Concentração
Correcta - que são a mente equânime e equilibrada, a serenidade emocional. A
serenidade é onde as emoções são equilibradas, apoiando"-se umas às outras, não
têm altos nem baixos. Existe uma sensação de felicidade, de serenidade; perfeita
harmonia entre o intelecto, os instintos e as emoções. Suportam"-se e ajudam"-se
mutuamente. Deixam de estar em conflito ou de nos levar a extremos. Por essa
razão, surge uma tremenda paz nas nossas mentes. Há uma sensação intrépida e de
à vontade que provém do Caminho Óctuplo - uma sensação de equanimidade de
equilíbrio emocional. Sentimos bem"-estar em vez daquela sensação de ansiedade,
tensão e conflito emocional. Temos clareza, temos felicidade, serenidade e sapiência.
Esta compreensão do Caminho Óctuplo deve ser cultivada, isto é \emph{bhāvanā}.
Usamos esta palavra para significar desenvolvimento.

\section{Aspectos da Meditação}

Esta mente reflexiva ou equilíbrio emocional é desenvolvido com base na prática
da meditação e respectivas técnicas de concentração e de estado consciente. Por
exemplo, durante um retiro podem experimentar passar uma hora a praticar
meditação \emph{samatha} em que apenas concentram a mente num objecto, digamos a
sensação da respiração. Continuam a trazê"-lo à consciência e a mantê"-lo de modo a
adquirir uma presença contínua na mente.

Deste modo, estão a mover"-se na direcção daquilo que se está a passar no vosso
próprio corpo em vez de serem puxados para o exterior, para os objectos dos
sentidos. Se não tiverem nenhum refúgio no interior estão constantemente a sair,
a serem absorvidos em livros, comida e todo o tipo de distracções. Mas este
imparável movimento da mente é muito cansativo. Assim, em vez disso, a prática
torna"-se na observação da respiração, que significa terem de se escusar a
seguirem a tendência de procurar algo fora. Têm de trazer a atenção para a
respiração do próprio corpo e concentrar a mente nessa sensação. À medida que
largam a forma grosseira estão na realidade a tornar"-se nessa sensação, no
próprio sinal do objecto. Durante um certo período de tempo tornam"-se naquilo 
no qual se absorvem. Quando realmente se concentram tornam"-se nessa mesma condição
tranquilizadora. Realizaram a tranquilidade. A meditação \emph{samatha} é esse
processo de se tornarem algo. 

Mas se investigarem essa tranquilidade percebem que não é satisfatória. Falta"-lhe 
algo, pois está dependente de uma técnica e de se estar apegado a algo que tem um 
princípio e um fim. Aquilo em que se tornam é temporário, pois a mudança, como a 
própria palavra sugere, é algo que se altera, uma condição impermanente. Não é a 
derradeira realidade. Não importa o quanto se avance na concentração, será sempre uma
condição insatisfatória. A meditação \emph{samatha} leva a grandes e radiantes
experiências na mente, mas todas elas terminam.

Depois, se praticarem meditação \emph{vipassanā} por mais uma hora, estando
totalmente presentes, largando tudo e aceitando a incerteza, o silêncio e a
cessação das condições, o resultado é que se tornarão pacíficos em vez de
simplesmente tranquilos. E~essa é a paz perfeita, completa. Não é como a
tranquilidade da meditação \emph{samatha} que possui, mesmo no seu auge, algo
imperfeito e insatisfatório. Quanto à realização da cessação, quando se
desenvolve e se compreende melhor, adquire"-se verdadeira paz e não apego,
Nibbāna.

Assim, \emph{samatha} e \emph{vipassanā} são as duas divisões na meditação.
Desenvolvemos estados de mente concentrados em que a consciência se torna
refinada através dessa mesma concentração. Mas ser imensamente refinado, possuir
um grande intelecto e gosto pela grandiosa beleza, faz com que tudo aquilo que é
grosseiro se torne intolerável, devido ao apego àquilo que é refinado. As
pessoas que dedicaram a vida somente ao refinamento e ao requinte acham a vida
terrivelmente frustrante e assustadora quando deixam de poder manter padrões tão
elevados.

\section{Racionalidade e Emoção}

Se admiram o pensamento racional e se são apegados a ideias e percepções, há a
tendência para desprezar as emoções. Podem observar esta tendência, sempre que
ao começarem a sentir emoções disserem, “Vou apagá"-las da mente. Não quero
sentir essas coisas”. Vocês sabem que se não sentirem nada poderão entrar numa
elevada vibração, causada pela pureza da inteligência e pelo prazer do
pensamento racional. A mente aprecia a sua forma lógica e controlada, a forma
como consegue fazer sentido. É simplesmente límpida, organizada e precisa como a
matemática, mas as emoções estão sempre todas dispersas e confusas, não é?
Não são precisas, não são organizadas e pode"-se facilmente perder o controlo.

Deste modo, a natureza emocional é frequentemente desprezada. Temos medo dela.
Por exemplo, nós homens frequentemente manifestamos medo das emoções porque ao
crescermos o que nos ensinam e no qual somos levados a acreditar, é que os homens
não choram. Em criança, pelo menos na minha geração, ensinavam"-nos que os
meninos não choram e assim tentávamos viver de acordo com os padrões daquilo que
os meninos deveriam ser. Diziam"-nos, “Tu és um homem”, e assim tentávamos ser
aquilo que os nossos pais nos diziam que deveríamos ser. As ideias da sociedade
afectam as nossas mentes e, por causa disso, consideramos as emoções
embaraçosas. Em Inglaterra, as pessoas geralmente acham as emoções embaraçosas;
se alguém fica um pouco mais emocionado, assume"-se que deve ser italiano ou de
qualquer outra nacionalidade.

Se se é muito racional e se tem tudo bem delineado, fica"-se sem saber o que
fazer quando as pessoas se emocionam. Se alguém começa a chorar, pensa"-se, “O
que é que devo fazer?”. Talvez se diga, “Anima"-te! Está tudo bem, meu caro. Vai
ficar tudo bem, não há motivo para chorar”. Se alguém é muito apegado ao
pensamento racional tem a tendência de, com a lógica, ignorar as emoções. Mas as
emoções não respondem à lógica, muitas vezes reagem, mas não respondem. A emoção é
algo muita sensível e opera de uma forma que por vezes não compreendemos.
Se nunca tentámos compreender o que realmente significa sentir a vida, e
contudo, se nos permitirmos ser sensíveis, tudo aquilo que é emocional parece"-nos
muito assustador e embaraçoso. Não percebemos de que se trata, pois
rejeitamos esse aspecto de nós próprios.

No meu trigésimo aniversário, compreendi que era um homem emocionalmente
subdesenvolvido. Foi um aniversário importante para mim, compreendi que era um
homem feito e maduro. Já não me considerava um jovem, mas  penso que por vezes 
reagia emocionalmente como se tivesse seis anos de idade; a esse nível não me
tinha desenvolvido muito. Ainda que conseguisse manter a pose e a presença de um
homem maduro na sociedade, nem sempre me sentia dessa forma. Ainda havia na
minha mente sentimentos e medos muito fortes por resolver. Pareceu"-me que teria
de fazer algo, pois a ideia de ter de passar o resto da minha vida ao nível
emocional de seis anos era uma triste perspectiva.

É aqui que muitos de nós nesta nossa sociedade ficam encalhados. A sociedade
americana, por exemplo, não permite que nos desenvolvamos emocionalmente, que
amadureçamos. Não compreende mesmo essa necessidade e por esse motivo não
fornece quaisquer ritos de passagem para os homens. A sociedade não fornece esse
tipo de introdução ao mundo da maturidade - deve"-se ser imaturo para toda a
vida. Deve"-se agir com maturidade, mas ninguém espera que sejamos maduros.

Por essa razão poucas pessoas o são. Na realidade, as emoções não são
compreendidas ou resolvidas, as tendências infantis são meramente suprimidas em
vez de melhoradas.

\enlargethispage{-\baselineskip}

O que a meditação faz é oferecer uma oportunidade de amadurecimento no plano
emocional. Perfeita maturidade emocional seria \emph{sammā vāyāma}, \emph{sammā
  sati} e \emph{sammā samādhi}. Isto serve para reflexão; não encontrarão isto
em nenhum livro, é para contemplarem. A perfeita maturidade emocional inclui
Esforço Correcto, Consciência Correcta e Concentração Correcta. Está presente quando
não estamos envolvidos em flutuações e vicissitudes, quando temos equilíbrio e
clareza e somos capazes de ser sensíveis e receptivos.

\section{As Coisas Tal Como São}

Com Esforço Correcto podemos aceitar as diferentes situações com calma, em vez
de entrar em pânico por pensar que depende de nós pôr toda a gente na linha,
tornar tudo certo e resolver os problemas de todos. Fazemos o melhor que
podemos, mas compreendemos que não depende só de nós fazer tudo isto ou tornar
tudo melhor.

A dada altura, quando estava em Wat Pah Pong com o Ajahn Chah, apercebi"-me de
muitas coisas que estavam erradas no mosteiro. Então fui ter com Ajahn Chah e disse,
«Ajahn Chah, isto não está a correr bem; tem de fazer alguma coisa». Ele olhou
para mim e disse, «Oh, sofres tanto, Sumedho. Tu sofres tanto. Tudo isso
mudará». E eu pensei, «Ele não quer saber! Este é o mosteiro ao qual ele dedicou
a vida e está a deixar ir tudo pelo cano abaixo!». Mas ele tinha razão. Passado
algum tempo a situação começou a mudar e, somente pelo facto de aguentar a
situação calma e pacientemente, as pessoas começaram a perceber o que estavam a
fazer. Às vezes, temos de permitir que vá tudo pelo cano abaixo para que as
pessoas possam perceber essa experiência.

Percebem o que quero dizer? Por vezes certas situações na nossa vida são assim,
não há nada que possamos fazer, por isso permitimos que as coisas sejam como são;
ainda que se tornem piores, permitimos que assim seja. Porém, ao fazermos isso
não estamos a reagir de forma negativa ou fatalista; é um tipo de paciência:
estar disposto a aguentar algo permitindo que a situação mude naturalmente, em
vez de egoisticamente tentar aprumar e limpar tudo, por não gostarmos ou termos
aversão à confusão.

Então, quando as pessoas nos perturbam, nem sempre nos ofendemos, magoamos ou
ficamos transtornados com o que acontece, nem ficamos despedaçados ou destruídos
com aquilo que nos possam dizer ou fazer. Conheço uma pessoa que exagera tudo.
Se hoje algo corre mal, ela diz, “Estou totalmente destruída!”, quando o que
aconteceu foi apenas um pequeno problema. No entanto, a mente dela exagera"-o de
tal forma, que uma coisa insignificante pode destrui"-la completamente para o resto
do dia. Quando percebemos isto, devemos compreender que estamos perante um
grande desequilíbrio, pois pequenas coisas não deviam puder despedaçar alguém por
completo.

Compreendi que me posso ofender facilmente e por isso fiz um voto de que jamais
me iria ofender. Tinha notado como era fácil ficar ofendido com pequenas
coisas, quer fossem intencionais ou não. Podemos ver como é fácil alimentar a dor,
a mágoa, a ofensa, a tristeza ou a preocupação; é como se algo existisse em nós
que está sempre a querer ser simpático, mas que sempre se sente um pouco
ofendido com isto ou um pouco magoado com aquilo.

Com reflexão, pode"-se ver que o mundo é assim; é um lugar sensível. Nem sempre
vos vai confortar e fazer"-vos sentir felizes, seguros e positivos. A vida está
repleta de coisas que podem ofender, magoar, ferir ou despedaçar. Isto é a vida.
Ela é assim. Se alguém falar num tom de voz mais exaltado, claro que o vão
sentir, mas depois a mente pode continuar a repetir e ficar ofendida: “Magoou"-me
mesmo quando ela disse aquilo; sabes, não foi um tom de voz muito agradável.
Senti"-me muito magoado. Nunca fiz nada para a magoar”. A mente continua a
repetir"-se desta forma, não é? Fica"-se despedaçado, magoado ou ofendido! Mas
então, se contemplarem, percebem que é somente sensibilidade.

Quando contemplam desta forma, não quer dizer que estejam a querer não sentir.
Quando alguém nos fala num tom de voz desagradável não é que não o sintamos. Não
estamos a querer ser insensíveis, mas antes a tentar não o interpretar de forma
errada, a não o tomar de forma pessoal. Ter emoções equilibradas significa que
as pessoas nos podem dizer coisas ofensivas, e nós saberemos aceitá"-las. Significa termos
equilíbrio e força emocional para não nos ofendermos, ferirmos ou despedaçarmos
com aquilo que acontece na vida.

Alguém que esteja sempre a sentir"-se magoado ou ofendido, tem de passar a vida a
fugir e a esconder"-se ou então, tem de encontrar um grupo de lambe"-botas
subservientes com quem possa viver; pessoas que dizem: “Você é maravilhoso,
Ajahn Sumedho”. “Será que sou mesmo maravilhoso?” “É pois”. “Está a dizer isso
só por dizer, não é?” “Não, não! Acredite, é do fundo do coração”. “Bom, aquela
pessoa não acredita que eu seja maravilhoso”. “Ora, ele é estúpido!” “Isso foi o
que eu pensei”.

É como a história “O Rei vai nu”, não é? Têm de procurar ambientes especiais, em
que tudo seja do vosso agrado, seguro e sem quaisquer ameaças.

\clearpage

\section{Harmonia}

Quando existe Esforço Correcto, Consciência Correcta e Concentração Correcta,
tornamo"-nos intrépidos. Somos intrépidos porque não há nada de que ter medo.
Temos a coragem de ver as coisas e de não as interpretar de forma errada; temos
a sabedoria para contemplar e reflectir sobre a vida; temos a segurança e
confiança de \emph{sīla}, a força do nosso compromisso moral e a determinação de
fazer o bem e evitar fazer o mal, seja através de acções ou de palavras. Desta forma,
todas as peças se encaixam formando o caminho para o desenvolvimento. É um
caminho perfeito porque tudo ajuda e apoia; o corpo, a natureza emocional –
sensibilidade do sentimento – e a inteligência. Todos estão em perfeita
harmonia, apoiando"-se mutuamente.

Sem essa harmonia o nosso instinto natural pode tornar"-se disperso e confuso. Se
não tivermos nenhum compromisso moral os nossos instintos podem tomar o
controlo. Por exemplo, se apenas seguirmos os desejos sexuais sem qualquer
referência moral, tornamo"-nos prisioneiros de todo o tipo de coisas que causam
aversão pessoal. Existe adultério, promiscuidade e doenças, e toda a perturbação
e confusão que provém de não frearmos o nosso instinto natural, com as
limitações da moralidade.

Podemos usar a nossa inteligência para enganar e mentir, certo? Mas quando temos
uma estrutura moral somos guiados pela sabedoria e pelo \emph{samādhi}; estes
conduzem ao equilíbrio e força emocional. Mas não usamos sabedoria para suprimir
a sensibilidade. Não dominamos as nossas emoções com o pensamento ou suprimindo
a sua natureza. Esta tem sido a nossa tendência no Ocidente; usamos os nossos
ideais e pensamentos racionais, para dominar e suprimir as nossas emoções, e
assim tornarmo"-nos insensíveis para com a vida e para connosco.

No entanto, através da meditação \emph{vipassanā} e de praticarmos estar conscientes,
a mente fica totalmente receptiva e aberta, possuindo as qualidades de plenitude
e total aceitação. E porque fica aberta, a mente também se torna reflexiva.
Quando se concentram num ponto a mente deixa de ser reflexiva, fica absorvida
na qualidade desse objecto. A capacidade reflexiva da mente que surge ao cultivarmos
um estado consciente produz uma mente mais abrangente e descondicionada. 
Uma mente que não está constantemente a filtra e a selecciona nada, mas simplesmente a 
constatar que tudo aquilo que  surge cessa. Sabe que as coisas às quais há apego, também 
irão cessar. Experiencia"-se a atractividade que ainda existe quando algo surge, mas com 
a noção de que tudo mudará, rumo à cessação, momento no qual essa atracção diminuirá. 
Teremos então de encontrar algo mais nos qual nos absorveremos\ldots{}

A questão de se ser humano é que temos de tocar a terra, ter os pés bem assentes
no chão, temos de aceitar as limitações desta forma física e desta vida
planetária. O caminho para sair do sofrimento não se encontra no abandono da
nossa experiência humana, vivendo em refinados estados de consciência, mas sim no
abraçar a totalidade do reino humano e dos \emph{Brahmas} através da consciência. Desta forma, o Buddha indicou o caminho para a realização total, em vez
de uma fuga momentânea através do refinamento e da beleza. Isto é o que o Buddha
quer dizer quando aponta o caminho para o Nibbāna.

\section{O Caminho Óctuplo como Ensinamento Reflectivo}

Neste Caminho Óctuplo, os oito elementos são como oito pernas a suportar"-nos.
Não é como: 1, 2, 3, 4, 5, 6, 7, 8 numa escala linear, mas sim como um trabalho
em grupo. Não é que primeiro desenvolvam \emph{paññā} e só depois quando há
\emph{paññā}, desenvolvem \emph{sīla}; e que uma vez o \emph{sīla}
desenvolvido, adquirem \emph{samādhi}. É assim que pensamos, não é?

“Obtém"-se o um, depois o dois e depois o três”. Como realização propriamente
dita, desenvolver o Caminho Óctuplo é uma experiência num momento. Todas as
partes trabalham em conjunto para o seu desenvolvimento; não é um processo
linear, podemos pensar que assim é porque só podemos ter um pensamento de cada
vez.

Tudo o que disse acerca do Caminho Óctuplo e das Quatro Nobres Verdades é apenas
uma reflexão. O que é verdadeiramente importante é que realmente percebam o que
estou a fazer quando reflicto, em vez de se apegarem àquilo que estou a dizer. É
um processo de trazer o Caminho Óctuplo à mente, usando"-o como ensinamento
reflectivo para que possam considerar o que realmente significa. Não pensem que o
compreendem apenas porque sabem explicar, “\emph{Sammā diṭṭhi} significa
Entendimento Correcto, \emph{Sammā saṅkappa} significa Pensamento Correcto”.
Isto é entendimento intelectual. Alguém pode dizer, “Não, eu penso que
\emph{sammā saṅkappa} significa\ldots{} ” e alguém responde, “Não, no livro diz
Pensamento Correcto. Tu estás errado”. Isso não é reflexão.

Podemos traduzir \emph{sammā saṅkappa} como Pensamento Correcto ou Atitude ou
Intenção; experimentamos diferentes \mbox{significados}. Podemos utilizá"-los como
ferramentas para a contemplação, em vez de pensarmos que são absolutamente
fixos, e que temos de os aceitar num estilo ortodoxo no qual qualquer tipo de variação
da interpretação exacta é heresia. Por vezes as nossas mentes pensam dessa forma
rígida, mas estamos a tentar transcender essa maneira de pensar, desenvolvendo
uma mente que se move, observando, investigando, considerando, questionando e
reflectindo.

Estou a tentar encorajar cada um de vós a serem suficientemente corajosos, para
sensatamente, considerarem a natureza da vida, em vez de terem alguém a
dizer"-vos se estão ou não preparados para a iluminação. Na realidade, o
ensinamento budista fala"-nos sobre ser"-se iluminado, aqui e agora, ao invés de
se fazer algo para nos tornarmos iluminados. A ideia de que se tem de fazer algo
para se atingir a iluminação só pode ter origem na comprensão incorrecta.
Dessa forma, a iluminação seria apenas mais uma condição dependente de outra, o que
não é realmente iluminação. Isso é somente uma percepção da iluminação. Todavia, não
estou a falar de nenhum tipo de percepção mas de como estarmos alerta perante a
natureza da vida. O momento presente é a única coisa que realmente podemos
observar: ainda não podemos observar o amanhã, e o ontem é só uma memória. Mas a
prática budista é muito directa, aqui e agora, olhando para as coisas como elas
são. Ora bem, como é que fazemos isso? Primeiro temos de olhar para as nossas
dúvidas e medos, pois tornamo"-nos tão apegados às nossas opiniões que as mesmas
nos levam a duvidar sobre o que estamos a fazer. Alguém pode desenvolver uma
confiança falsa e acreditar que é iluminado, mas acreditar que é ou não é
iluminado, é tudo ilusão.

Aquilo que estou a indicar é ser iluminado em vez de apenas acreditarmos que
somos iluminados, e para isso temos de nos abrir à verdade.

Começamos com as coisas como elas são neste preciso momento, tal como a
respiração do nosso próprio corpo. O que é que isso tem a ver com a Verdade, com
a iluminação? Será que observar a minha respiração significa que sou iluminado?
Mas quanto mais pensarem e tentarem perceber o que é, mais
incertos e inseguros se sentirão.

Tudo o que podemos fazer nesta forma física é abandonar a ilusão. Essa é a
prática das Quatro Nobres Verdades e o desenvolvimento do Caminho Óctuplo.

