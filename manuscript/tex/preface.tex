\chapter{Prefácio}

Este livro foi compilado e editado a partir de palestras proferidas pelo
Venerável Ajahn Sumedho sobre o ensinamento essencial do Buddha - que a
infelicidade humana pode ser transcendida através do caminho espiritual.

A primeira exposição das Quatro Nobres Verdades foi apresentada pelo Buddha, em
528 a.C., no Parque dos Veados em Sarnāth, perto de Varanāsi, através do
discurso \emph{Sutta Dhammacakkappavattana} – que literalmente
significa “o discurso que coloca em movimento o veículo do ensinamento”.
Excertos deste \emph{sutta} são citados no início de cada capítulo, descrevendo
as Quatro Nobres Verdades. Cada referência corresponde à secção dos livros das
escrituras (Pāli Cânon), onde este discurso pode ser encontrado. No entanto, nas
escrituras, o tema das Quatro Nobres Verdades repete-se algumas vezes, como por
exemplo na citação que aparece no início da Introdução.

Em muitas das suas palestras Ajahn Sumedho usa a expressão budista de “not-self”
(\emph{anattā}) “não eu”. Ajahn Sumedho ensina que a raiz da ignorância é a
ilusão da existência de um eu. Desta forma, não está a falar de aniquilação ou
da rejeição das qualidades pessoais, mas sim a indicar como o sofrimento
(\emph{dukkha}) surge quando querermos manter esta identificação com o corpo e
com a mente, sendo esta identificação errada, naquilo a que a maioria das
pessoas chama de “eu”.

Outro termo usado muitas vezes por Ajahn Sumedho nas suas palestras é
“deathless”, que surge neste livro com alguma frequência. Por não existir em
português uma única palavra que ilustre claramente o seu significado, foi
traduzido de diferentes formas, usando-se os termos que melhor se adequavam ao
contexto de cada situação. Podemos ainda acrescentar que a palavra se refere,
não ao sentido de imortalidade mas sim àquilo que está para além do ciclo de
vida e de morte, não em termos metafísicos mas sim no sentido de impermanência -
“Tudo o que surge está sujeito a cessar” – não se tratando portanto da
derradeira realidade. Nas escrituras existe uma passagem que pode ajudar a
clarificar um pouco mais a palavra “deathless”:

\begin{quote}
  «Existe, bhikkhus, o não nascido, o não formado, o não criado, o não
  originado. Se não existisse o não nascido, o não formado, o não criado, o não
  originado, não existiria o nascido, o formado, o criado e o originado. Porém,
  precisamente porque existe o não nascido, o não formado, o não criado e o não
  originado, a emancipação do nascido, formado, criado e originado é realizada».

  \quoteRef{Nibbāna Sutta, Ud 8.3}
\end{quote}

Luang Por Sumedho oferece-nos a seguinte reflexão sobre esta profunda
declaração: «Podemos ver que não somos vítimas prisioneiras da condição do
nascimento, sem qualquer esperança de escaparmos ao sofrimento da mudança, dos
nossos hábitos e desejos. Existe, assim, uma saída: realizar a existência do não
nascido, não formado, não criado, não originado. Reconhecer, isso é \emph{sati
  sampajaññā}, \emph{sati paññā} ou plena atenção. É perceber a diferença entre
estar, e não estar apegado à forma, ao que é criado. Nibbāna é a realidade do
não-apego aos fenómenos condicionantes; não se trata de destruir o
\emph{saṃsāra}, de aniquilar todos as condições por estas serem tão limitadoras
e só conduzirem ao sofrimento, mas sim reconhecer e discernir essa realidade».

Concluindo, estas Quatro Nobres Verdades são como que um exercício de
discernimento; ajudam a não tomar posições rígidas a favor ou contra o que quer
que seja, mas sim a reconhecer o Não Nascido e Não Criado como verdadeiro, e não
como uma fantasia ou um ideal. Desta forma, esta realidade é reconhecida e
cultivada na nossa vida quotidiana.

\bigskip

Caro leitor, o desejo é que, ao explorar as seguintes páginas, os corações de todos aqueles que tiveram a oportunidade
de encontrar a sabedoria dos ensinamentos aqui contidos, se
sintam inspirados a despertar, e rapidamente realizem o fim
de todo o sofrimento.

\bigskip

{\raggedleft
  Kāñcano Bhikkhu\\
  Mosteiro Amarāvatī\\
  Outubro 2007
\par}

