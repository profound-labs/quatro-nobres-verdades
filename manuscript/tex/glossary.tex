\chapter{Glossário}

\begin{glossarydescription}

% Ajahn – palavra Tailandesa para mestre, mentor, professor; frequentemente usada como titulo para monges seniores.
% Palavra equivalente ao Pāli “ācariya”.
% Bhikkhu – mendicante; termo para o monge budista que
% vive das ofertas e assume os preceitos de treino que definem
% a vida de renúncia e moralidade.
% Buddha rÆpa – (rūpa), forma, corpo. Corpo do Buddha
% ou Bodhi.
% Dhamma – (em Sânscrito: Dharma) esta palavra tem
% vários significados, tais como dever, a lei da verdade universal,
% a natureza ou constituição das coisas, lei, norma, objecto da
% mente, fenómenos ou princípios de comportamentos que ao
% serem seguidos integram os seres humanos na ordem natural
% das coisas; qualidades da mente a ser desenvolvidas para se
% poder compreender a qualidade da mente em si mesma. Nos
% textos a palavra é encontrada com todos estes significados.
% Dhamma refere-se tanto aos ensinamentos do Buddha contidos
% nas escrituras, como à experiência directa da verdade suprema
% para a qual os seus ensinamentos são direccionados.
% Dia de Observância – (em Pāli: Uposatha) um dia sagrado ou “sabbath”, ocorre em todos dias de Lua Nova e Lua
% Cheia. Nestes dias os budistas reúnem-se para ouvir o
% Dhamma e reafirmam a sua prática em termos de preceitos e
% meditação.
% 
% Kamma – (em Sânscrito: karma) - Acção de causa e
% efeito. Causa que é criada e recriada pelos impulsos habituais,
% vontade própria ou energias naturais. Manifesta-se de forma
% benéfica ou prejudicial no corpo, na linguagem e na mente.
% Marcas ou impressões que ficam na nossa mente causando o
% renascimento e moldando o destino dos seres. Este termo,
% popularmente usado, inclui também o sentido de resultado ou
% efeito da acção, embora o termo correcto para isto seja vipāka.
% Paticcasamuppāda – a apresentação por etapas de como
% o sofrimento surge dependendo do grau de ignorância e de
% desejo e, de como termina com a sua cessação.
% Tipitaka – literalmente “três cestos”, a colecção das
% escrituras Budistas, classificadas de acordo com Sutta
% (Discursos), Vināya (Disciplina ou Treino) e Abhidhamma
% (Metafísica).
% Vipassanā – significa ver a verdadeira natureza das coisas,
% na sua realidade. É um processo de auto-transformação
% através da auto-observação. Um método analítico baseado na
% atenção plena, vigilância e investigação dos fenómenos manifestados nos cinco agregados “khandha”, nomeadamente
% forma física “rūpa”, sensações ou sentimentos “vedanā”, percepção “saññā”, formações mentais “sankhāra” e consciência
% “viññāna”.

% === A ===

\item[anicca] (Pali) Impermanence: one of the \emph{three characteristics of
    existence} along with not-self (\emph{anattā}) and unsatisfactoriness
  (\emph{dukkha}).

% === B ===

\item[borapet] (Thai) Tinospora crispa. Heart-shaped moonseed or guduchi.
  An extremely bitter vine used as a prophylactic and treatment for malaria.

% === C ===

% === D ===

% === E ===

% === F ===

% === G ===

% === H ===

% === I ===

% === J ===

% === K ===

% === L ===

% === M ===

% === N ===

% === O ===

% === P ===

% === Q ===

% === R ===

% === S ===

% === T ===

% === U ===

% === V ===

% === W ===

\end{glossarydescription}

