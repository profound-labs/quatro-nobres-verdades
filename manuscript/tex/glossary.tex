\chapter{Glossário}

\thispagestyle{bottomcenter}

\begin{glossarydescription}

% === A ===

\item[Ajahn] (Thai) palavra Tailandesa para mestre, mentor, professor;
  frequentemente usada como título para monges seniores. Palavra equivalente ao
  Pāli “ācariya”.

% === B ===

\item[Bhikkhu] (Pāli) mendicante; termo para o monge budista que vive das
  oferendas e assume as regras de treino que definem a vida de renúncia e
  moralidade.

% === C ===

% === D ===

\item[Dhamma] (Pāli; em Sânscrito: \emph{Dharma}) esta palavra tem vários
  significados, tais como dever, a lei da verdade universal, a natureza ou
  constituição das coisas, lei, norma, objecto da mente, fenómenos ou princípios
  de comportamentos que ao serem seguidos integram os seres humanos na ordem
  natural das coisas; qualidades da mente a ser desenvolvidas para se poder
  compreender a qualidade da mente em si mesma. Nos textos a palavra é
  encontrada com todos estes significados. Dhamma refere"-se tanto aos
  ensinamentos do Buddha contidos nas escrituras, como à experiência directa da
  verdade suprema para a qual os seus ensinamentos são direccionados.

\item[Dia de Observância] (em Pāli: \emph{Uposatha}) um dia sagrado ou
  “sabbath”, ocorre em todos dias de Lua Nova e Lua Cheia. Nestes dias os
  budistas reúnem"-se para ouvir o Dhamma e reafirmam a sua prática em termos de
  disciplina e meditação.

% === E ===

% === F ===

% === G ===

% === H ===

% === I ===

% === J ===

% === K ===

\item[Kamma] (Pāli; em Sânscrito: \emph{karma}) Acção de causa e efeito. Causa que é
  criada e recriada pelos impulsos habituais, vontade própria ou energias
  naturais. Manifesta"-se de forma benéfica ou prejudicial no corpo, na linguagem
  e na mente. Marcas ou impressões que ficam na nossa mente causando o
  renascimento e moldando o destino dos seres. Este termo, popularmente usado,
  inclui também o sentido de resultado ou efeito da acção, embora o termo
  correcto para isto seja vipāka.

% === L ===

% === M ===

% === N ===

% === O ===

% === P ===

\item[Paṭicca-samuppāda] (Pāli) génese dependente, apresentação por etapas de
  como o sofrimento surge em função do grau de ignorância e de desejo e de
  como termina com a cessação destes.

% === Q ===

% === R ===

% === S ===

% === T ===

\item[Tipiṭaka] (Pāli) literalmente “três cestos”, a colecção das escrituras
  Budistas, classificadas de acordo com Sutta (Discursos), Vināya (Disciplina ou
  Treino) e Abhidhamma (Metafísica).

% === U ===

% === V ===

\item[Vipassanā] (Pāli) significa ver a verdadeira natureza das coisas, na sua
  realidade. É um processo de auto"-transformação através da auto"-observação. Um
  método analítico baseado num estado consciente, na vigilância e investigação dos
  fenómenos manifestados nos cinco agregados “\emph{khandha}”, nomeadamente
  forma física “\emph{rūpa}”, sensações “\emph{vedanā}”,
  percepção “\emph{saññā}”, formações mentais “\emph{saṅkhāra}” e consciência
  “\emph{viññāṇa}”.

% === W ===

\end{glossarydescription}

