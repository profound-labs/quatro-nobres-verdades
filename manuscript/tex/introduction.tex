\chapter{Introdução}

\thispagestyle{bottomcenter}

\begin{quote}
  «A razão porque, quer Eu, quer vocês, viajámos e deambulámos durante muito
  tempo neste longo ciclo, deve"-se a não termos descoberto nem penetrado quatro
  verdades. Quais são?

  São: A Nobre Verdade do Sofrimento, A Nobre Verdade da Origem do Sofrimento, A
  Nobre Verdade do Cessar do Sofrimento e A Nobre Verdade do Caminho que conduz
  à Cessação do Sofrimento».

  \quoteRef{Dīgha Nikāya 16}
\end{quote}

\noindent
O \emph{Sutta Dhammacakkappavattana}, o ensinamento do Buddha sobre as Quatro
Nobre Verdades, tem sido a principal referência que tenho usado na minha prática
ao longo dos anos.

\enlargethispage{\baselineskip}

É o ensinamento que usávamos no nosso mosteiro na Tailândia.
A escola Budista Theravada, considera este \emph{sutta} como a quinta"-essência
dos ensinamentos do Buddha. Este \emph{sutta} contém tudo o que é necessário
para compreender o Dhamma e alcançar a iluminação.

Apesar de o \emph{Sutta Dhammacakkappavattana} ser considerado como o primeiro
sermão dado pelo Buddha após a sua iluminação, por vezes gosto de pensar que ele
deu o seu primeiro sermão quando encontrou aquele asceta a caminho de Varanāsi.
Depois da sua iluminação em Bodh Gaya, o Buddha pensou: «Trata"-se de um
ensinamento tão subtil. Não conseguirei de modo algum expressar por palavras
aquilo que descobri e por isso não o ensinarei. Permanecerei sentado debaixo da
árvore Bodhi para o resto da minha vida».

Para mim esta é uma ideia bastante tentadora, desaparecer simplesmente e viver
sozinho e não ter de lidar com os problemas da sociedade. No entanto, enquanto o
Buddha pensava, Brahma Sahampati (a divindade criadora no Hinduísmo) apareceu e
convenceu o Buddha de que ele deveria partir e ensinar. Brahma Sahampati
disse"-lhe que existiam seres que iriam compreender, seres que tinham somente um pouco
de poeira nos olhos. Assim o ensinamento do Buddha foi dirigido àqueles com um
pouco de poeira nos olhos – Tenho a certeza de que ele não pensou que o
ensinamento se tornaria num movimento tão vasto.
Depois da visita de Brahma Sahampati, o Buddha segue o seu caminho de Bodh Gaya
para Varanāsi, quando encontra um asceta que fica impressionado com a sua
aparência tão radiante. O asceta pergunta"-lhe «O que é que tu descobriste?» e o
Buddha responde: «Eu sou o perfeitamente iluminado, o \emph{Arahant}, o Buddha».

\enlargethispage{\baselineskip}

Gosto de considerar este como sendo o seu primeiro sermão. Foi um fracasso
porque o homem que o ouviu, pensou que o Buddha tivesse praticado demais e se
estivesse a sobrevalorizar. Se alguém nos dissesse estas palavras, tenho a
certeza que reagiríamos da mesma forma. O que é que fariam se eu dissesse, «Eu
sou o perfeitamente iluminado?».

Na verdade, a declaração do Buddha foi um ensinamento muito correcto e preciso.
É o ensinamento perfeito, mas nós somos incapazes de o compreender, devido a
pensar e a interpretar erroneamente, que uma afirmação como esta provém do ego,
já que as pessoas entendem tudo sobre o ponto de vista dos seus próprios egos.
«Eu sou o perfeitamente iluminado» pode soar como uma declaração egóica, mas não
é na verdade puramente transcendental? É interessante reflectirmos nesta declaração: 
«Eu, o Buddha, o perfeitamente iluminado», porque ela liga o uso de
“Eu sou” com realizações e conquistas supremas. De qualquer forma, o resultado
do primeiro ensinamento do Buddha, foi que o ouvinte nada conseguiu compreender
e continuou no seu caminho.

\enlargethispage{\baselineskip}

Mais tarde, o Buddha encontrou os seus antigos companheiros no Parque dos
Cervos, em Varanāsi. Os cinco eram sinceramente dedicados ao ascetismo severo.
Eles tinham ficado desiludidos com o Buddha, pois pensavam que ele já não era
sincero na sua prática, uma vez que, antes da sua iluminação, tinha começado a
perceber que o ascetismo austero não conduzia ao estado de iluminação e deixando assim essa prática. Os cinco amigos pensaram que era desleixo - talvez o tenham
visto a comer arroz de leite, o que hoje em dia, pode ser comparado a comer um
gelado. Se fossem ascetas e vissem um monge a comer gelado talvez perdessem a fé
nele, por pensarem que os monges só devem comer sopa de urtigas.

Se gostassem mesmo de ascetismo e me vissem a comer uma taça de gelado,
deixariam de ter fé em Ajahn Sumedho. É assim que funciona a mente humana;
prefere admirar grandes feitos de auto"-flagelação e renúncia.

Quando os cinco amigos e discípulos perderam a fé no Buddha, deixaram"-no – o que
lhe deu a oportunidade de se sentar debaixo da árvore Bodhi para alcançar a
iluminação.

Mais tarde, quando encontraram o Buddha no Parque dos Cervos em Varanāsi,
pensaram, «Sabemos bem como ele é. Não vale a pena ligar"-lhe». Mas quando o
Buddha se aproximou, todos sentiram que havia nele algo especial. Levantaram"-se
para lhe dar lugar e ele então proferiu o sermão das Quatro Nobres Verdades.

Desta vez, em vez de dizer «Eu sou o iluminado», ele disse: «Existe sofrimento.
Existe a origem do sofrimento. Existe a cessação do sofrimento. Existe o caminho
para a cessação do sofrimento». Apresentado desta forma, o seu ensinamento não
necessita de aceitação ou rejeição. Se ele tivesse dito «Eu sou o todo
iluminado», seríamos forçados a concordar, a discordar ou até ficarmos confusos.
Não saberíamos bem como interpretar tal afirmação. No entanto, dizendo: «Existe
sofrimento, existe uma causa, existe um fim para o sofrimento e existe o caminho
para a cessação do sofrimento», ele ofereceu algo para reflexão: «O que é que se
quer dizer com isto? O que é que se quer dizer com sofrimento, a sua origem, a
cessação e o caminho?».

\enlargethispage{\baselineskip}

Assim começamos a observar, a pensar. Com a afirmação: «Eu sou o todo
iluminado», talvez apenas discutíssemos: «Será que ele é realmente
iluminado?\ldots{} » «Eu penso que não» – não estamos preparados para um ensinamento
tão directo. Obviamente, o primeiro sermão do Buddha falhou porque foi
transmitido a alguém que ainda tinha bastante poeira nos olhos. Assim, na
segunda oportunidade, ele proferiu o sermão das Quatro Nobres Verdades.

\sectionBreak

As Quatro Nobres Verdades são: existe sofrimento, existe uma causa ou origem
para o sofrimento, existe a cessação do sofrimento e existe um caminho para
abandonar o sofrimento, que é o Óctuplo Caminho. Cada uma destas Verdades é
constituída por três fases, perfazendo assim um total de doze revelações. Na
escola Theravada, o “\emph{Arahant}”, o purificado, é alguém que claramente
assimilou as Quatro Nobres Verdades com as suas três fases e doze revelações.
“\emph{Arahant}” significa um ser humano que compreende verdadeiramente o
ensinamento das Quatro Nobres Verdades.

Na Primeira Nobre Verdade, “Existe sofrimento” é a primeira revelação. Qual é o
significado dessa revelação? Não necessitamos de vê"-lo como algo grandioso,
trata"-se apenas de reconhecer que “Existe sofrimento”. Esta é uma revelação
básica. A pessoa ignorante diz, «Estou a sofrer. Não quero sofrer. Eu medito e
vou a retiros para deixar de sofrer, mas continuo a sofrer e não quero
mais\ldots{} Como é que posso sair deste sofrimento? O que é que posso fazer
para me ver livre dele?». Mas isto não é a primeira Nobre Verdade pois esta não
se trata de “Existe sofrimento e eu quero pôr"-lhe fim”. A revelação é “Existe
sofrimento”.

Assim, há que observar a dor e angústia que se sente, não do ponto de vista de
“Isto é meu”, mas como uma reflexão: “Existe este sofrimento, este
\emph{dukkha}”. Tal vem a partir da posição reflectiva de “Buddha observando o
Dhamma”. A revelação é simplesmente o reconhecimento, de que o sofrimento existe
sem se tornar pessoal. Esse reconhecimento é uma revelação importante;
simplesmente observar a angústia da mente ou da dor física e vê"-las como
\emph{dukkha} em vez de infortúnio pessoal, não reagindo às mesmas da forma
habitual.

A segunda revelação da primeira Nobre Verdade é: “O sofrimento deve ser
compreendido”. A segunda revelação ou, aspecto de cada uma das Nobres Verdades,
contém nela a \mbox{palavra} “deve”: “Deve ser compreendido”. Assim a segunda revelação
diz"-nos que \emph{dukkha} é algo para ser compreendido. Em vez de nos querermos livrar
de \emph{dukkha}, devemos \mbox{compreendê"-lo}.

Apesar de “compreender” ser uma palavra bastante vulgar, em Pāli significa
aceitar verdadeiramente o sofrimento, acolhê"-lo em vez de reagir. Com qualquer
forma de sofrimento, quer seja físico ou mental, geralmente só reagimos; mas com
compreensão podemos realmente observar o sofrimento, aceitá"-lo e abraçá"-lo
verdadeiramente. “Devemos compreender o sofrimento” é então a segunda revelação
da Primeira Nobre Verdade.

A terceira revelação da Primeira Nobre Verdade é: “O sofrimento foi
compreendido”. Quando realmente se vive o sofrimento, observando"-o, aceitando"-o,
percebendo"-o e deixando"-o ser da forma que é, temos então, a terceira revelação:
“O sofrimento foi compreendido” ou “\emph{Dukkha} foi compreendido”. Assim,
estes são os três aspectos da Primeira Nobre Verdade: “Existe \emph{dukkha}”
“Deve ser compreendido” e “Foi compreendido”.

\sectionBreak

Este é o padrão para as três fases de cada Nobre Verdade. Primeiro temos a
declaração, depois a receita e por fim o resultado da prática. Podemos também
defini"-lo em termos do seu significado em Pāli, \emph{pariyatti},
\emph{patipatti} e \emph{pativedha}. \emph{Pariyatti} é a teoria ou declaração:
Existe sofrimento”. \emph{Patipatti} é a prática, mais propriamente praticar com
a declaração e \emph{pativedha} é o resultado da prática. Isto é o que chamamos
de padrão de reflexão, uma vez que conduz ao desenvolvimento da mente de uma
forma mais profunda. A mente búdica é uma mente reflexiva que conhece as coisas
como elas realmente são.

Usamos estas Quatro Nobres Verdades para o nosso desenvolvimento. Aplicámo"-las a
coisas comuns na nossa vida, aos mais vulgares apegos e obsessões da mente. Com
estas verdades podemos investigar os nossos apegos e obsessões para obtermos as
revelações. Através da Terceira Nobre Verdade, podemos realizar a cessação, o
fim do sofrimento e praticar o Caminho Óctuplo, até obtermos compreensão.
Quando o Caminho Óctuplo tiver sido plenamente desenvolvido somos um
\emph{Arahant} – tarefa cumprida. Embora isto possa parecer complicado – quatro
verdades, três fases e doze revelações - é bastante simples. É uma ferramenta
que usamos para nos auxiliar a compreender o que é, e o que não é o sofrimento.

No mundo budista são poucos os que ainda usam as Quatro Nobres Verdades, até
mesmo na Tailândia. As pessoas dizem, “Ah sim, as Quatro Nobres Verdades –
coisas de principiante”. Talvez até usem todos os métodos de \emph{vipassanā} e
se tornem realmente obcecados com as dezasseis etapas antes de chegarem às
Nobres Verdades. Eu acho realmente espantoso, que no mundo budista o ensinamento
verdadeiramente mais profundo tenha sido posto de parte, como sendo Budismo
primitivo: “Isso é para os miúdos pequenos, os principiantes. O curso superior
é\ldots{}”. E partem para complicadas ideias e teorias, esquecendo o mais
profundo ensinamento.

As Quatro Nobres Verdades são uma reflexão para a vida inteira. Não se trata
apenas de realizar as Quatro Nobres Verdades, as três fases e doze revelações e
assim alcançar o estado de \emph{Arahant}, num único retiro, e então partir para
algo mais avançado. As Quatro Nobres Verdades não são assim tão fáceis.
Necessitam de uma constante atitude de vigilância e oferecem"-nos pretexto para
uma vida de investigação.

